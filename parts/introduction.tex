%https://github.com/tenhobi/bachelors-thesis/issues/11
\begin{introduction}
Poslední desetiletí jsou mobilní telefony nedílnou součástí všedního dne.
Telefony používáme pro práci, komunikaci na sociálních sítích i zábavu.
Obchody s aplikacemi obsahují obrovské množství aplikací a her,
které si může každý stáhnout během několika chvil.
Velkou část aplikací lze stáhnout i bezplatně,
případně lze bezplatně stáhnout alespoň určitou část hry.

Podle statistik a trendů z herního odvětví \cite{wepc_video_games} lze vidět,
že na světě je přes 2,5 miliardy hráčů.
77 \% uživatelů mobilních zařízeních jsou hráči,
přičemž v roce 2013 to bylo pouhých 52,3 \% uživatelů.
76 \% lidí preferuje hrát hry na mobilních zařízeních,
oproti 62\% lidí, kteří preferují hraní na PC.
Průměrný věk hráčů je 33 let, průměrný věk hráček je 37 let.
92 \% her dostupných na obchodě Google Play bylo zdarma ke stažení.

Z těchto trendů vyplývá motivace pro vývoj právě mobilní hry,
která je zaměřena především na mladší část publika,
je dostupná zdarma a je snadná na pochopení i hraní,
což umožní poskytnout hru více potencionálním hráčům.

Tato práce se zaměřuje na analýzu, návrh, implementaci a testování
multiplatformní mobilní herní aplikace s prvky kooperace.
Aplikace je určena pro převážně mladší hráče
a je navžena dostatečně jednoduše,
aby byla nenáročná a tím více atraktivní.

Práce je rozdělena do 7 kapitol.
V první kapitole je rozebrán výběr čtyř konkurenčních aplikací,
včetně zhodnocení kladů a záporů jednotlivých her.
Tato rešerše pomáhá k inspiraci pro vytvoření lepšího zážitku pro navrhovanou
hru.
V druhé kapitole jsou popsány možnosti technologií,
které lze využít pro praktickou část práce.
Jednoltivé typy technologií jsou na závěr každé podkapitoly zhodnoceny
a jsou vysvětleny procesy, které vedly k výběru právě dané technologie.
Ve třetí kapitole je provedena analýza hry, požadavků a případů užití.
Čtvrtá kapitola je zaměřena na návrh uživatelského rozhraní a architektiry.
Obě části mají detailněji rozebrány zajímavé vybrané části při návrhu.
Pátá kapitola se zabývá popisem použitých nástrojů
a jednotlivých přínosů, který daný nástroj přináší.
Jsou popsány použité knihovny, které kvalitně řeší dílčí části.
A v neposlední řadě je představeno několik vybraných zajímavých
implementačních částí.
Šestá kapitola je zaměřena na testování.
V kapitola jsou popsány použité typů testů,
uživatelské testování a především jsou představeny vybrané ukázky testů
implementace.
V poslední, sedmé, kapitole se čtenář dozvídá o způsobech nasazení a
možnostech distribuce aplikace pro mobilní operační systémy Android a iOS.

V závěru práce je pak zhodnocení výsledků práce a možnosti rozšíření.
Důležitou součástí jsou také splněné cíle.
\end{introduction}
