\begin{conclusion}
Cílem této bakalářské práce bylo analyzovat, navrhnout a implementovat
funkční kooperativní multiplatformní mobilní herní aplikaci. 

V teoretické části práce byla provedena rešerše konkurenčních aplikací
a možností technologií pro multiplatformní framework, state management,
databáze, senzory, architekturu a testování.
Každá technologie čítala několik možností pro výběr technologie
a na základě výhod a nevýhod jednotlivých možností byla vybrána vždy
jedna možnost.

V rámci analýzy byly popsány principy hry.
Principy byly podrobně analyzovány a na základě toho byly vytvořeny
požadavky a následně případy užití,
které popisují nekolik scénářů průchodů aplikace.

V praktické části práce bylo vycházeno z analýzy
a byl vytvořen návrh uživatelského rozhraní
a architektury aplikace.
Další kapitola se zabývá implementací,
kde byly uvedeny zajímavé ukázky kódu
a konkrétní implementace stěžejních částí aplikace.
Aplikace byla poté implementována a otestována testy,
které jsou automatizované a jsou spouštěny například v rámci průběžné integrace.

Byla vytvořena uživatelská a vývojářská dokumentace.
Uživatelská dokumentace vhodně provádí uživatele všemi scénáři průchodu
aplikací a možnostmi nastavení. 

Všechny cíle této práce byly splněny.
Aplikaci lze navíc v budoucnu rozšířit mnoha způsoby,
ať už vylepšením samotného obsahu hry, herních misí, dostupnými moduly,
tak i vylepšením o zcela nové funkce.
Příkladem možných funkcí mohou být například
přidání podpory pro webové prohlížeče,
využití nových typů senzorů a práce s nimi,
nebo rozšíření hry o mód s více hráči.
\end{conclusion}
