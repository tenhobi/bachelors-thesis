\begin{conclusion}
Cílem této bakalářské práce bylo analyzovat, navrhnout a~implementovat
kooperativní multiplatformní mobilní herní aplikaci. 

Na základě cílů je v~teoretické části práce provedena rešerše
konkurenčních aplikací
a~možností technologií pro multiplatformní framework, správu stavů,
databáze, senzory, architekturu a~testování.
Po rešerši technologií,
která čítá popis několika možností pro výběr technologie,
byly vybrány jako nejlepší možnosti
multiplatformní framework Flutter,
knihovna pro správu stavů Bloc,
databáze Cloud Firestore
a~architektura Clean Architecture.

V~rámci analýzy byly popsány principy hry.
Principy byly podrobně analyzovány a~na základě toho byly vytvořeny
požadavky a~následně případy užití,
které popisují nekolik scénářů průchodů aplikace.

Dle rešerše a~analýzy bylo navrženo uživatelské rozhraní,
které dbá na jednoduchý a~srozumitelný vzhled,
a~architektura aplikace.
Další kapitoly se zabývájí popisem implementace a~testování,
kde byly uvedeny zajímavé ukázky kódu
a~konkrétní implementace částí aplikace.
Aplikace byla implementována a~otestována testy,
které jsou automatizované a~jsou spouštěné například v~rámci průběžné integrace.

Na základě implementace byla vytvořena uživatelská a~vývojářská dokumentace.
Uživatelská dokumentace vhodně provádí uživatele všemi scénáři průchodu
aplikací a~možnostmi nastavení.

Všechny cíle této práce byly splněny.
Aplikaci lze navíc v~budoucnu rozšířit mnoha způsoby,
ať už vylepšením samotného obsahu hry, herních misí, dostupnými moduly,
tak i~vylepšením o~zcela nové funkce.
Příkladem možných funkcí mohou být například
přidání podpory pro webové prohlížeče,
využití nových typů senzorů a~práce s~nimi,
nebo rozšíření hry o~mód s~více hráči.
\end{conclusion}
