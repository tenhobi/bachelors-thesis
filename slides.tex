% arara: xelatex: { shell : yes }

% kompilace xelatex prezentace.tex
% dokumentace k beameru: http://ftp.cvut.cz/tex-archive/macros/latex/contrib/beamer/doc/beameruserguide.pdf

\RequirePackage[hyphens]{url}

% nastavení formátu prezentace 16:9 
\documentclass[czech,aspectratio=169]{beamer}

\usepackage{polyglossia}
\setmainlanguage{czech}

% nastavení vzhledu 
% další možnosti vzhledu viz https://hartwork.org/beamer-theme-matrix/
\usetheme{Madrid}
\usecolortheme{whale}

% vzhled slajdů vnitřní téma (např. vzhled odrážek)
\useinnertheme{rectangles} %možnosti: default circles rectangles rounded inmargin
% vzhled slajdů vnější téma
\useoutertheme{default} %možnosti: default, miniframes, smoothbars, sidebar, split, shadow, tree, smoothtree, infolines

% zavedeme čvutí modou barvu
\definecolor{CVUT}{HTML}{0065BD}
% čvutí modou použijeme jako hlavní barvu prezentace
\setbeamercolor{structure}{bg=white,fg=CVUT}

% jako font prezentace nadefinujeme oficiální ČVUT písmo Technika -- pokud chcete použít, musíte si font nainstalovat nebo jej nahrát na Overleaf
% https://www.cvut.cz/logo-a-graficky-manual  -- inforek, přihlášení přes celoškolské heslo
%\usepackage{fontspec}
%\setsansfont{Technika-Kniha}

% vypneme navigační panel beamer (pro zapnutí zakomentujeme)
%\beamertemplatenavigationsymbolsempty

% vygenerujeme slajdy s poznámkami -- ty si můžete vytisknout a mít je na obhajobu s sebou (pokud zapomenete slova, nebo kdyby nefungovalo promítání z nějakého důvodu)
%\setbeameroption{show notes}

% další balíčky
\usepackage{graphicx}
\usepackage{minted}
\usepackage{hyperref}
\usepackage{tikz}
\usetikzlibrary{chains,fit,shapes}

% Údaje o prezentaci
\title[Kooperativní mobilní multiplatformní hra]{Kooperativní mobilní multiplatformní hra}
\subtitle{Bakalářská práce}
\institute[FIT ČVUT v~Praze]{Fakulta informačních technologií \\ České vysoké učení technické v~Praze}
\author[J. Bittner]{Jan Bittner \\ Vedoucí práce: Ing. Marek Suchánek}
\date{23. 6. 2020 TODOTODOTODO}
\titlegraphic{\includegraphics[width=.1\textwidth]{assets/slides/logo-cvut}}

\begin{document}
  \begin{frame}
    \titlepage 
    \note{Nezapomenout pozdravit} %tohle je poznámka, ta na slajdu nebude, ale vygeneruje se vedle něj, pokud odkomentujete příkaz výše -- \setbeameroption{show notes} 
  \end{frame}
  
  %\begin{frame}
  %  \tableofcontents %generuje se automaticky z section, subsection, subsubsection
  %\end{frame}

  \begin{frame}{Motivace volby tématu}
    \begin{center}
      77 \% uživatelů mobilních zařízeních hraje hry.
      \vskip5mm
      Vytvoření mobilní hry pro více platforem\\
      ve formě open-source řešení\\
      se zaměřením na snadnou rozšiřitelnost a~udržitelnost.
    \end{center}
  \end{frame}

  \begin{frame}{Cíle práce}
    \begin{itemize}
      \item analýza podobných aplikací a~trendů
      \item analýza technologií pro vývoj
      \item návrh hry a~herní logiky
      \item návrh a~implementace hry
      \item otestování funkčnosti
    \end{itemize}
  \end{frame}

  \begin{frame}{Technologie}
    \begin{itemize}
      \item \textbf{Flutter} -- multiplatformní framework
      \item \textbf{Bloc} -- knihovna pro správu stavů
      \item \textbf{Clean Architecture} -- architektura
      \item \textbf{Cloud Firestore} -- NoSQL databáze
    \end{itemize}
  \end{frame}

  \begin{frame}{Clean Architecture}
    \begin{center}
      \includegraphics[width=.6\textwidth]{assets/slides/logo-clean-architecture}
    \end{center}
  \end{frame}

  \begin{frame}
      \begin{center}
        {\large ``The way you keep software soft is
      to leave as many options open as possible,
      for as long as possible.
      What are the options that we need to leave open?\\
      They are the details
      that don’t matter.''}
      \vskip5mm
      --- Robert C. Martin, Clean Architecture
      \end{center}
  \end{frame}

  \begin{frame}{Návrh hry}
    \begin{center}
      \includegraphics[width=.9\textwidth]{assets/slides/event-loop}
    \end{center}
  \end{frame}

  \begin{frame}
    \begin{columns}
      \begin{column}{.25\textwidth}
        \begin{center}
          \includegraphics[width=.9\textwidth]{assets/slides/screen-0}
        \end{center}
      \end{column}
      \begin{column}{.25\textwidth}
        \begin{center}
          \includegraphics[width=.9\textwidth]{assets/slides/screen-1}
        \end{center}
      \end{column}
      \begin{column}{.25\textwidth}
        \begin{center}
          \includegraphics[width=.9\textwidth]{assets/slides/screen-2}
        \end{center}
      \end{column}
      \begin{column}{.25\textwidth}
        \begin{center}
          \includegraphics[width=.9\textwidth]{assets/slides/screen-3}
        \end{center}
      \end{column}
    \end{columns}
  \end{frame}

  \begin{frame}{Shrnutí}
    \begin{itemize}
      % TODO -- doladit
      %\item Rešerše vhodných technologií.
      %\item Návrh a vytvoření funkčního řešení hry.
      \item Všechny cíle splněny.
      \item Úspěšný návrh a~vytvoření funčního řešení hry.
      \item Úspěšné cílení na snadnou rozšířitelnost.
      \item Přínos aplikace pro obohacení komunity Flutter.
      \item Spokojenost uživatelů při testování aplikace.
      \item Mnoho možností rozšíření a~adaptace.
    \end{itemize}
  \end{frame}

  \begin{frame}{Zdroje}
    \begin{itemize}
      \item WEPC. \emph{2020 Video Game Industry Statistics, Trends \& Data} [online]. 2020 [cit. 2020-03-21]. Dostupné z: \url{https://www.wepc.com/news/video-game-statistics}.
      \item MARTIN, Robert C. \emph{The Clean Architecture} [online]. 2012 [cit. 2020-04-07]. Dostupné z: \url{https://blog.cleancoder.com/uncle-bob/2012/08/13/the-clean-architecture.html}.
      \item MARTIN, Robert C. \emph{Clean Architecture: A~Craftsman’s Guide to Soft-ware Structure and Design}. Prentice Hall, 2018, s. 140. ISBN 0134494164. Dostupné z: \url{https://www.amazon.com/Clean-Architecture-Craftsmans-Software-Structure-ebook/dp/B075LRM681}. 
    \end{itemize}
  \end{frame}

  \begin{frame}{Děkuji za pozornost}
    \begin{center}
      Prostor pro dotazy.
    \end{center}
  \end{frame}

  \begin{frame}[noframenumbering]{Otázky oponenta}
    Otázka první: Proč?

    \vfill

    Odpověď: Prostě proto.
  \end{frame}

  \begin{frame}[noframenumbering]{Otázky oponenta}
    Otázka druhá: Proč?

    \vfill

    Odpověď: Prostě proto.
  \end{frame}
\end{document}
