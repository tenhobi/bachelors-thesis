\section{Keep Talking and Nobody Explodes}

\begin{figure}
    \centering
    \includegraphics[width=0.5\linewidth]{assets/competitive-apps/keep-talking.jpg}
    \caption{Screenshot hry Keep Talking and Nobody Explodes~\cite{steelcrategamesinc_keep_talking}}
    \label{fig:keep-talking}
\end{figure}

Ve hře \emph{Keep Talking and Nobody Explodes} závisí všechno na správné
kooperaci a komunikaci.
Tato velmi oceňovaná hra existuje ve verzi pro desktop, konzole,
virtuální realitu i mobily~\cite{steelcrategamesinc_keep_talking}.

Hra má předpřipravená herní kola s daným typem bomby.
Bomba je koncipována jako kufřík s několika moduly a časomírou,
která se nebezpečně rychle odpočítává.
Jeden hráč vidí bombu.
Další hráč, případně skupina hráčů, mají k dispozici manuál s instrukcemi.
Manuál mohou prohlížet jak v elektronické, tak i v papírové podobě.
Manuál je totiž pro všechny hry shodný.

Veškerá komunikace probíhá pouze slovně.
Hráč s bombou tak nahlásí hráči s manuálem například to,
že vidí červené tlačítko s textem \uv{defuse}.
Druhý hráč pak v manuálu vyhledá daný modul tlačítka, řídí se instrukcemi a
podstatné informace k zneškodnění předá prvnímu hráči.

Cíl každého kola je zneškodnění bomby.
Důraz je také kladen na čas zneškodnění a počet udělaných chyb.
Po skončení kola může hráč porovnat svůj výsledek s ostatními hráči v žebříčku.

\subsection*{Klady}

\begin{itemize}
    \item UI je laděné jako pohled na stůl, na kterém je bomba a časomíra,
    což dodává hře atmostféru a vtáhne hráče do děje.
    \item Světla v místnosti blikají, případně se náhodně vypínají a zapínají.
    To přidává na efektu zejména v situácích,
    kdy zbývá už jen pár sekund a bomba není dobře vidět.
    \item Přestože je grafika jednodušší, samotná bomba je velmi přehledná.
    \item Manuál je stejný pro všechna herní kola. Lze jej tedy i vytisknout.
\end{itemize}

\subsection*{Zápory}

\begin{itemize}
    \item U složitějších modulů je manuál občas hůře pochopitelný.
\end{itemize}
