\chapter{Implementace}
\label{chap:implementation}

Tato kapitola se zabývá popisem implementace aplikace.
V~následujících podkapitolách jsou popsány použité nástroje,
použité knihovny a~vybrané ukázky.
Aplikace byla implementována dle návrhu v~kapitole~\ref{chap:design}.

\section{Použité nástroje}

Během vývoje byly použity nástroje,
které usnadnily vývoj aplikace či její ladění.
Pro samotné ladění aplikace byl využit nástroj \emph{scrcpy},
který umožní promítat obrazovku mobilního zařízení na obrazovku počítače
včetně bezdrátového spojení.

\begin{listing}
    \caption{Spuštění nástroje scrcpy pro bezdrátové použití}
    \label{code:scrcpy}
    \begin{minted}{shell}
adb tcpip 5555
adb connect DEVICE_IP:5555
scrcpy --window-title "Totally Not Chernobyl" --always-on-top
--bit-rate 2M --max-size 800
    \end{minted}
\end{listing}

\subsection{Verzování}

Během vývoje bylo využito verzovacích nástrojů.
To je běžná součást každého projektu a~slouží k~postupnému zaznamenávání změn.
V~této práci byl využit verzovací systém \emph{Git} a~webová služba \emph{GitHub},
která nabízí kromě bezplatného umístění repozitáře také doplňkové služby jako
code coverage, ticket systém, posouzení změn a~podobně.

\subsection{Editor}

Během vývoje aplikace byly využity dva editory.
Editor IntelliJ IDEA od společnosti JetBrains
a~editor VS Code od společnosti Microsoft.
Editor IntelliJ IDEA je robustní editor,
který obsahuje nápovědu a~analýzu kódu.
Díky výhodám jazyka Dart a~frameworku Flutter
byla však i~práce v~editoru VS Code snadná.
Pro oba editory jsou dostupné v~editorech chytré doplňky,
které využívají integrovaných nástrojů dodávaných s~jazykem Dart,
ale i~věci navíc jako poskytování útržků kódu pro usnadnění práce.

\section{Použité knihovny}

Při implementaci bylo využito několik knihoven,
které elegantně řeší dané specifické problémy.
Mezi nejzajímavější použité knihovny patří knihovny
\mintinline{Dart}|bloc| a~\mintinline{Dart}|flutter_bloc|,
které poskytují rozhraní knihovny a~widgety pro práci s~nimi.
Knihovna \mintinline{Dart}|easy_localization| řeší překlad uživatelského
rozhraní aplikace do několika jazyků.
Uživatel mezi jazyky může přepínat a~tím měnit jazyk prvků.
Lokalizace může být řešena více způsoby,
nejjednoduší a~jeden z~nejvíce efektivních je pomocí vytvoření několika souborů,
například ve formátu \mintinline{shell}|JSON| nebo \mintinline{shell}|YAML|,
které jednoduše a~přehledně umožní definovat slovník překladu.
Samotná definice ovšem ne vždy stačí.
Knihovna umožňuje vkládat parametry, které se do daného překladu vloží.
Knihovna \mintinline{Dart}|google_sign_in| poskytuje jednoduché rozhraní pro
přihlašování pomocí účtu Google.
Jelikož jazyk Dart nepodporuje třídy union,
knihovna \mintinline{Dart}|freezed| poskytuje rozhraní a~generátor pro
snadné generování těchto tříd.
Knihovna \mintinline{Dart}|url_launcher| umožní spouštění webových odkazů
nehledě na platformu.
Knihovna \mintinline{Dart}|cloud_firestore| poskytuje rozhraní k~práci se
stejnojmennou databází Cloud Firestore.
Dále je využito knihovny \mintinline{Dart}|effective_dart|,
která poskytuje souhrn pravidel pro statickou analýzu dle konvencí jazyka Dart.

\section{Dokumentace}

Z~implementace byla vytvořena také automatická vývojářská dokumentace.
Ta je ve formě webové stránky a~zobrazuje veřejné rozhraní tříd.
Komentáře jsou přehledně zobrazeny pomocí nástroje Markdown a~obsahují proklik
na třídy a~metody.
Tato dokumentace je umístěna na přiloženém datovém CD.

\section{Ukázky vybraných částí}

V~následujících sekcích je uvedeno několik ukázek z~implementace.
Každá ukázka reprezentuje určitou část vývoje a~popisuje její zajímavosti.

\subsection{Implementace widgetu}

V~ukázce kódu~\ref{code:implementation-1} je uvedena reprezentace implementace widgetu
ve frameworku Flutter.
Implementován je widget pro zobrazení loga aplikace.
Jedná se o~stateless widget,
což znamená,
že takovýto widget je zobrazen staticky a~nic se v~něm nemění.
Widget může mít několik vstupních proměnných,
které využívá,
a~dokonce může využívat i~další widgety,
které mohou být stateful.

Jak lze z~ukázky vidět,
je využito standardních widgetů jako je widget \mintinline{Dart}|Column|,
který slouží pro uspořádání prvků do sloupce pod sebe
a~widgetu \mintinline{Dart}|Text|,
který slouží pro výpis textu.
Oba widgety mají řadu parametrů,
které náležitě modifikují jejich styl.
V~ukázce je také vidět,
že je využita metoda \mintinline{Dart}|tr()|,
která slouží pro vyžádání překladu pomocí knihovny \mintinline{Dart}|easy_localization|.

Zajímavostí jazyka Dart také je,
že umožňuje vkládat do seznamu podmíněné položky,
jak lze vidět ve spodní části kódu.
To umožňuje tvořit více čtivý kód a~redukuje množství kódu.

\begin{listing}
    \caption{Implementace widgetu ve frameworku Flutter}
    \label{code:implementation-1}
    \begin{minted}{Dart}
class AppLogo extends StatelessWidget {
    final bool withVersion;
    final Brightness brightness;

    AppLogo({
    Key key,
    this.withVersion = true,
    this.brightness = Brightness.light,
    }) : super(key: key);

    @override
    Widget build(BuildContext context) {
        return Column(
            crossAxisAlignment: CrossAxisAlignment.center,
            mainAxisAlignment: MainAxisAlignment.center,
            children: <Widget>[
            Column(
                crossAxisAlignment: CrossAxisAlignment.end,
                children: <Widget>[
                Text(
                    tr('title'),
                    textAlign: TextAlign.end,
                    style: GoogleFonts.ubuntu(
                    fontSize: 32,
                    color: brightness == Brightness.light
                        ? Colors.white
                        : Colors.black,
                    ),
                ),
                if (withVersion == true)
                  Version(brightness: brightness),
                ],
            ),
            ],
        );
    }
}
    \end{minted}
\end{listing}

\subsection{Datová třída}

S~využitím knihovny \mintinline{Dart}|freezed| je možné tvořit nejen
třídy union,
ale také datové třídy.
Tyto třídy po vygenerování obsahují kopírující konstruktor,
metody pro porovnání rovnosti,
metody pro převod do formátu JSON a~podobně.
Práce s~takovouto třídou je tedy velmi snadná.

\begin{listing}
    \caption{Implementace datové třídy}
    \label{code:implementation-2}
    \begin{minted}{Dart}
@freezed
abstract class User with _$User {
    const factory User({
        @required String username,
        @required String language,
        @Default(0) int gamesCount,
        @Default(0) int winGamesCount,
    }) = _User;

    factory User.fromJson(Map<String, dynamic> json) =>
      _$UserFromJson(json);
}
    \end{minted}
\end{listing}

%\subsection{\todo{Ukázka 3}}
