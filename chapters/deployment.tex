\chapter{Sestavení a distribuce aplikace}

Pro testování výsledků aplikace je nutné aplikaci spustit.
Pro spuštění je však nutné aplikaci sestavit.
Sestavení aplikace na vývojáře obstará framework,
zatímco vývojář se musí ujistit,
že správně nainstaloval a nakonfiguroval příslušné nástroje.
Před samotným sestavením aplikace je nutné provést kroky k nastavení prostředí
a příslušných nástrojů:

\begin{enumerate}
    \item Nainstalovat programovací jazyk Dart
    --- tento krok nemusí být vyžadován,
    avšak některé externí nástroje s ním pracují.
    \item Nainstalovat framework Flutter.
    \item Nainstalovat balíčky Dart a Flutter pro dané IDE.
    \item Nainstalovat Android Studio pro platformu Android
    a Xcode pro platformu iOS.
    \item Je-li potřeba, nainstalovat emulátor Android či iOS zařízení.
\end{enumerate}

Správné nastavení lze ověřit spuštěním nástroje
\mintinline{shell}|flutter doctor|.
Po nastavení prostředí a příslušných nástrojů přichází na řadu nastavení služeb,
které aplikace využívá.
Konkrétně je nutné přidat konfigurační soubory pro zprovoznění služby Firebase.
Konfigurační soubor lze stáhnout z webové aplikace Firebase Console,
kde se také vytváří a konfiguruje příslušná databáze Cloud Firestore. 

Po tomto nastavení je možno aplikaci sestavit,
například pro účely debuggování.
Další kroky nastavení pro distribuce aplikace pro danou platformu jsou
zmíněny v dalších sekcích.
Aplikaci sestavíme pomocí příkazu \mintinline{shell}|flutter build|.
Je také možné aplikaci přímo spustit buď na daném emulátoru,
nebo na připojeném fyzickém zařízení.
V případě fyzického zařízení je vhodné se ujistit,
že je zařízení řádně připojeno a detekování,
pomocí příkazu \mintinline{shell}|flutter devices|.
Po úspěšném detekování zařízení lze aplikaci sestavit a spustit na vybraném
zařízení pomocí příkazu \mintinline{shell}|flutter run|. 

V následujících sekcích jsou popsány postupy k distribuci aplikace na
platformách Android a iOS.
Následně je také popsáno,
jak bude probíhat sestavení a distribuce na dalších potencionálních
platformách,
které budou přidány v průběhu dalšího vývoje frameworku Flutter.

\section{Android}

Pro sestavení aplikace na platformě Android je nutné stáhnout a~přidat soubor
\mintinline{shell}|android/app/google-services.json|,
který obsahuje nastavení služby databáze Cloud Firestore.
Aplikaci je vhodné přidat vlastní ikonu,
která se uloží do složky \mintinline{shell}|android/app/src/main/res/|.
Pro samotnou distribuci je nutné sestavit aplikaci
s~podepsaným certifikačním klíčem,
jinak aplikace nepůjde publikovat v~obchodě Google Play. 
Tento klíč se vygeneruje pomocí nástroje \mintinline{shell}|keytool|
a~je nutné na něj odkázat ze souboru \mintinline{shell}|android/key.properties|.
V~souboru
\mintinline{shell}|android/app/src/main/AndroidManifest.xml|
lze také nastavit položku
\mintinline{shell}|application android:label|
pro nastavení názvu aplikace.
V~souboru
\mintinline{shell}|android/app/build.gradle|
lze naopak nastavit údaje pro distribuci,
konkrétně položka \mintinline{shell}|applicationId|
obsahuje unikátní identifikátor aplikace
a~položky \mintinline{shell}|versionCode| a~\mintinline{shell}|versionName|
obsahují název
a~číslo konkrétního vydání aplikace.~\cite{flutter_deploy_android}

Příkaz \mintinline{shell}|flutter build appbundle| sestaví výslednou aplikaci
pro distribuci,
kde \emph{app bundle} je preferovaný formát pro distribuci,
oproti klasickému formátu \emph{apk}.
Následně je možné distribuovat aplikaci přes rozhraní Google Play Console
do obchodu Google Play.

\section{iOS}
 
Pro sestavení aplikace na platformě iOS je nutné stáhnout a přidat soubor
\mintinline{shell}|ios/Runner/GoogleService-info.plist|,
který obsahuje nastavení služby databáze Cloud Firestore.
V souboru \mintinline{shell}|ios/Runner/Info.plist|
je možné nastavit položku \mintinline{shell}|CFBundleName|
pro změnu názvu aplikace.
Ve stejném souboru lze také nastavit položky
\mintinline{shell}|CFBundleShortVersionString| a
\mintinline{shell}|CFBundleVersion|,
reprezentující název a číslo konkrétního vydání.
Nastavení názvů a dalších údajů,
včetně certifikátů a ID aplikace,
lze nastavit v programu Xcode.
Program Xcode také za vývojáře automaticky
spravuje certifikát.~\cite{flutter_deploy_ios}

Příkaz \mintinline{shell}|flutter build ios| sestaví výslednou aplikaci
pro distribuci.
Následně je možné distribuovat aplikaci do obchodu Apple Store.

\section{Další platformy}

Framework Flutter momentálně podporuje pouze platformy Android a~iOS.
Platformy jako je web a~desktop jsou dostupné pro testování.
Aplikace je díky univerzálnosti frameworku připravena k rozšíření i~na další
platformy~\cite{flutter_deploy_web}.
Stejně jako pro mobilní platformy,
i~tyto platformy budou muset nakonfigorovat databázi Cloud Firestore.
Potřeba konfigurace ostatních služeb závisí na potřebách jednotlivých balíčků.
Předpokladem je,
že části aplikace využívající například senzory,
budou muset být pro tyto platformy upraveny,
či~zcela vynechány,
jelikož platformy web ani desktop běžně používanými senzory nedisponují.
Platforma web se sestaví pomocí příkazu \mintinline{shell}|flutter build web|.
Sestavení je dostupné ve složce \mintinline{shell}|build/web/|
a~soubory jsou automaticky minifikovány.

