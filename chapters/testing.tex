\chapter{Testování}
\label{chap:testing}

Tato kapitola se zabývá popisem testování aplikace.
Aplikace byla testována jak při vývoji vývojářem,
tak i automatizovanými testy.
Bylo provedeno také uživatelské testování.

Díky vlastnostem frameworku Flutter bylo možné ladit a testovat aplikaci
při vývoji díky vlastnosti \emph{hot reload}.
Tato vlastnost při spuštění aplikace ve vývojářském řežimu
aplikuje změny do běžící aplikace,
avšak uchovává současný stav jednotlivých widgetů.
Díky tomu je možné okamžitě reagovat na změny a chyby v kódu
a nemusí se tak čekat na zdlouhavé sestavení.

Cílem automatizovaných testů je otestovat funkčnost jednotlivých částí aplikace
a otestovat příslušné rozhraní.
Automatizované testy lze spouštět ručně i automaticky,
například v rámci kontinuální integrace,
což je velmi praktické a bylo při vývoji hojně využíváno.
Během vývoje aplikace byla vytvořena řada unit testů,
které testují patřičné části aplikace.
Widget testy vytvořeny nebyly,
protože nemají takovou vypovídající hodnotu a jsou velmi náchylné na změny v UI.

Velký zřetel byl dán na testování použitelnosti.
Toto testování má za cíl odhalit jak chyby aplikace,
tak zejména samotnou srozumitelnost a jednoduchost průchodu aplikace.
Toto testování je dále popsáno v podkapitole~\ref{sec:usability}.

\section{Ukázky vybraných částí}

V následujících sekcích je uvedeno několik ukázek z testování.

\subsection{Získání uživatele}

V ukázce kódu~\ref{code:test-get-user} lze vidět ukázka testování získání
uživatele dle jeho ID.
Tento test testuje příslušný usecase pomocí metody mockování,
což je proces který imituje závislost kódu.
Pro daný test tak není podstatná implementace závislosti,
tedy jak je získán daný uživatel,
ale skutečnost,
že je využito správné metody repozitáře.

\begin{listing}
    \caption{Test získání uživatele dle id}
    \label{code:test-get-user}
    \begin{minted}{Dart}
GetUserByUid usecase;
MockUserRepository mockUserRepository;

setUp(() {
    mockUserRepository = MockUserRepository();
    usecase = GetUserByUid(mockUserRepository);
});

final tUid = 123;
final tUser = User(username: 'test', language: 'test');

test(
    'should get user for the uid from the repository',
    () async {
        // arrange
        when(mockUserRepository.getUserByUid(any))
            .thenAnswer((_) async => Right(tUser));
        // act
        final result = await usecase(Params(uid: tUid));
        // assert
        expect(result, Right(tUser));
        verify(mockUserRepository.getUserByUid(tUid));
        verifyNoMoreInteractions(mockUserRepository);
    },
);
    \end{minted}
\end{listing}

\subsection{Serializace}

V ukázce kódu~\ref{code:test-user-serialization} je testována správná
serializace entity uživatele.
V tomto testu je využito metody využívající takzvané fixtures,
což jsou předpřipravená data v souboru,
se kterými může test dále pracovat.
Díky tomu může test načíst soubor ve formátu JSON a převést jej do mapy v jazyce
Dart.
Tento výstup je následně porovnán s očekávaným výsledkem.
Tento test tedy testuje,
zda je ze souboru JSON správně vytvořen objekt entity.

\begin{listing}
    \caption{Test serializace}
    \label{code:test-user-serialization}
    \begin{minted}{Dart}
final tUser = User(
    username: 'test username',
    language: 'test language',
    gamesCount: 3,
    winGamesCount: 1,
);

group('fromJson', () {
    test(
        'should return a valid model when the JSON number'
        'is an integer',
        () async {
            // arrange
            final Map<String, dynamic> jsonMap =
                json.decode(fixture('user.json'));
            // act
            final result = User.fromJson(jsonMap);
            // assert
            expect(result, tUser);
        },
    );
});
    \end{minted}
\end{listing}

\section{Testování použitelnosti}
\label{sec:usability}

Testování použitelnosti testuje aplikaci z pohledu uživatelů.
Uživatel dostane několik úkolů popsaných scenáři,
které musí splnit a následně je porovnáván postup či samotný úspěch řešení.
Jednotlivé testovací scénáře jsou umístěny v příloze~\ref{chap:test-scenarios}.

Testování použitelnosti se zúčastnilo celkem 11 osob ---
osm mužů a tři ženy.
Deset respondentů bylo z věkové skupiny 18--26 let,
zbývajícímu respondentovi bylo 47 let.
Všichni respondenti měli mobilní zařízení s operačním systémem Android.

Testování proběhlo pomocí videohovorů za přítomnosti vývojáře a respondenta.
Každý respondent dostal sadu úkolů dle scénářů
z přílohy~\ref{chap:test-scenarios}
a byl seznámen s podstatou herní aplikace.
Tyto úkoly popsané scénáři mají za cíl najít či odhalit chyby nebo nedostatky
v aplikaci z pohledu uživatele.

\subsection{Průběh testování}

Scénář~\ref{scen:login} se zabývá přihlášením do aplikace.
Respondenti neměli s vyhledáním tlačítka přihlášení
a s vyplněním registrace žádný problém.
Obrazovka přihlášení a registrace je dle respondentů přehledná a srozumitelná.

Ve scénáři~\ref{scen:about} měli respondenti za úkol zobrazit informace o
aplikaci.
Tento úkol splnili všichni bez potíží a po stisku tlačítka \emph{O aplikaci}
se přesunuli bez komplikací na obrazovku s potřebnými informacemi.

Ve scénáři~\ref{scen:join} nejprve vývojář vytvoří hru a respondent má za úkol
se do této hry připojit pomocí obdrženého speciálního kódu.
Většina respondentů bez problémů našla tlačítko směřující na obrazovku připojení
do hry,
avšak tři respondenti měli menší obtíže dané tlačítko vyhledat.
Pro dané tlačítko byla proto zvolena vhodnější ikona.
Na zobrazené obrazovce neměl již žádný z respondentů problém vyplnit kód
a připojit se do místnosti.

Ve scénáři~\ref{scen:create} je obrácený úkol oproti předešlému scénáři.
Respondent má nyní za úkol vytvořit danou hru.
Najít příslušné tlačítko nebyl pro žádného respondenta problém
a respondenti hodnotili velmi kladně výraznost prvku.
Volba mise kapitoly a následné spuštění hry proběhlo bez komplikací.

Ve scénáři~\ref{scen:profile} měli respondenti za úkol zobrazit profil a
statistiky uživatele.
Oba úkoly proběhly bez komplikací.
Tlačítko pro přesun na příslušnou obrazovku našli všichni respondenti.

Ve scénáři~\ref{scen:logout} bylo úkol odhlášení se z aplikace.
Většina respondentů tento úkol správně splnila přechodem na obrazovku menu
a kliknutím na příslušnou ikonu tlačítka odhlášení.
Dva respondenti tlačítko neviděli,
avšak po menší pomoci i oni tlačítko úspěšně našli.

Výstup tohoto testování je hodnocen velmi kladně.
Respondenti aplikaci a její přehlednost hodnotili dobře a ve většině případů
neměli problém s navigací a vyhledáním potřebných informací.