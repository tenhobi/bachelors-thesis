\section{Funkční požadavky}

Funkční požadavky popisují jednotlivé požadavky na funkcionalitu aplikace.
Z~pohledu uživatelů aplikace popisují akce,
které může uživatel provést.
Jednotlivé funkční požadavky,
či jejich uskupení,
mohou tvořit jednotlivé nezávislé moduly aplikace.~\cite{fr_nfr}

\begin{enumerate}[label=\textbf{F\arabic*}, ref=F\arabic*]
    \myItem{Přihlášení}
Uživatel se bude moci do aplikace přihlásit, resp. registrovat,
pomocí účtu Google.
    \myItem{Informace o~aplikaci}
Aplikace obsahuje obrazovku s~informacemi o~aplikaci,
včetně jména autora, verze, popisu a~odkazu na projekt.
    \myItem{Vytvoření hry}
Přihlášený uživatel může vytvořit hru.
Současně hráč nastaví misi ke hraní.
    \myItem{Připojení do hry}
Přihlášený uživatel se může připojit do hry pomocí kódu.
    \myItem{Spuštění hry}
Po potvrzení, že jsou hráči připraveni na hru,
mohou spustit hru s~danou misí.
    \myItem{Ukončení hry}
Hráč může během přípravy hry nebo hraní hry danou hru ukončit.
    \myItem{Statistika po hře}
Hráčům je po skončení hry zobrazena herní statistika.
    \myItem{Zobrazení svého profilu}
Přihlášený uživatel si může zobrazit svůj profil.
    \myItem{Zobrazení svých statistik}
Přihlášený uživatel si může zobrazit své souhrnné statistiky
ze všech hraných misí.
    \myItem{Změna nastavení}\label{req:settings}
Přihlášený uživatel si může změnit nastavení.
    \myItem{Odhlášení}\label{req:logout}
Přihlášený uživatel se bude moci z~aplikace odhlásit.
\end{enumerate}
