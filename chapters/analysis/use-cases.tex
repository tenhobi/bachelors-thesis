\let\oldsubsection=\thesubsection
\renewcommand\thesubsection{UC\arabic{subsection}} % UC* counter for subsections

\section{Případy užití}

Funkční a nefunkční požadavky sice definují funkcionalitu a vlastnosti aplikace,
avšak pro správné pochopení fungování a souhru těchto požadavků,
je nutné definovat případy užití,
které popisují jednotlivé problémy.
Případy užití jsou vytvořeny na základě stanovených funkčních požadavků,
s ohledem na stanovené nefunkční požadavky.

\subsection{Přihlášení}
\label{uc:login}

Tento případ užití popisuje proces přihlášení do aplikace.
Přihlášení probíhá pomocí účtu Google.
Proces zahrnuje i registraci,
pokud je uživatel v aplikaci nový.

\begin{enumerate}
    \item Uživatel klikne na tlačítko \todo{\emph{Přihlásit se}}
    a následně proběhne přihlášení pomocí účtu Google.
    \item V závislosti na tom,
    jestli je uživatel již registrovaný,
    mohou nastat dvě situace:
    \begin{enumerate}
        \item pokud uživatel již byl registrovaný,
        je uživatel přihlášen a přesměrován na hlavní obrazovku;
        \item pokud uživatel ještě nebyl registrovaný,
        zobrazí se obrazovka,
        ve které vyplní své osobní údaje
        a následně je přihlášen a přesměrován na hlavní obrazovku.
    \end{enumerate}
\end{enumerate}

\subsection{Zobrazení informací o aplikaci}

Tento případ užití poposije proces zobrazení informací o aplikaci
i o herních modulech.

\begin{enumerate}
    \item Uživatel se nachází na hlavní obrazovce a klikne na tlačítko
    \todo{\emph{Informace}}.
    \item Uživatel je přesměrován na obrazovku s informacemi o aplikaci,
    kde jsou zobrazeny informace jako jsou jméno vývojáře, verze aplikace,
    odkaz na repozitář se zdrojovým kódem nebo nejčastěji kladené otázky.
    \item Pokud chce uživatel vyhledat informace o jednotlivých herních
    modulech,
    lze se přesměrovat na danou obrazovku kliknutím na tlačítko
    \todo{\emph{Informace o modulech}}.
\end{enumerate}

\subsection{Změna nastavení}

Tento případ užití popisuje změnu nastavení,
kde lze zejména nastavit jazyk aplikace.

\begin{enumerate}
    \item Uživatel se nachází na hlavní obrazovce a klikne na tlačítko
    \todo{\emph{Nastavení}}.
    \item Uživatel je přesměrován na obrazovku s nastavením.
    Zde může uživatel například přepínat jazyk aplikace.
\end{enumerate}

\subsection{Hra}

Tento případ užití popisuje proces vytvoření, spuštění a průběhu hry.
Případ užití je dělen do dvou scénářů.
První scénář popisuje uživatele,
který hru zakládá.
Druhý scénář popisuje uživatele,
který se do hry připojuje.

\subsubsection*{Scénář A -- zakládající uživatel}

Tento scénář popisuje případ užití pro uživatele,
který hru založí a hostuje.
Tento uživatel má také práva pro nastavení hry,
jako je kapitola a mise nebo role.

\begin{enumerate}
    \item Uživatel klikne na tlačítko \todo{\emph{Vytvořit hru}}
    a aplikace jej přesune na obrazovku místnosti.
    \item Zde uživatel vidí speciální kód,
    který předá spoluhráči.
    \item Uživatel vybere kapitolu,
    ve které vybere misi,
    a určí svou roli ve hře
    --- technik či navigátor.
    \item Uživatel klikne na tlačítko \todo{\emph{Jsem připraven}}.
    \item Až budou oba hráči připraveni ke hře,
    uživatel klikne na tlačítko \emph{AZ-5} a zahájí tak hru.
    \item Po skončení hry se uživateli zobrazí statistika ze hry.
\end{enumerate}

\subsubsection*{Scénář B -- připojující se uživatel}

Tento scénář popisuje případ užití pro uživatele,
který se připojuje do již vytvořené hry.
Tento uživatel nemá žádná práva pro nastavení.

\begin{enumerate}
    \item Uživatel klikne na tlačítko \todo{\emph{Připojit se do hry}}
    a aplikace jej přesuna na obrazovku místnosti,
    kde zadá kód hry.
    \item Uživatel klikne na tlačítko \todo{\emph{Jsem připraven}}.
    \item Po skončení hry se uživateli zobrazí statistika ze hry.
\end{enumerate}

\subsection{Ukončení hry}

Tento případ užití popisuje případy,
kdy a jak lze ukončit probíhající hru.

\subsubsection*{Scénář A -- dohrání}

Tento scénář popisuje standardní případ,
kdy hráč skončí hru dle předpokladů,
ať už úspěšně či neúspěšně.

\begin{enumerate}
    \item Uživatel se připojí do hry.
    \item Uživatel hraje hru.
    \item Uživatel dohraje hru, ať už úspěšně nebo neúspěšně.
    \item Aplikace zobrazí herní statistiky.
    \item Uživatel následně opustí hru kliknutím na tlačítko \emph{zpět}.
\end{enumerate}

\subsubsection*{Scénář B -- předčasné ukončení hry}

Tento scénář popisuje situaci,
kdy se jeden z hráčů rozhodne předčasně,
kvůli jakémukoli důvodu,
opustit hru a tím jí ukončit. 

\begin{enumerate}
    \item Uživatel se připojí do hry.
    \item Uživatel hraje hru.
    \item Uživatel klikne na tlačítko \todo{\emph{Ukončit hru}}
    a potvrdí dialog.
    \item Aplikace zobrazí hlášku o předčasném ukončení hry.
\end{enumerate}

\subsection{Zobrazení profilu a statistik}

Tento případ užití popisuje,
jak uživatel zobrazí svůj profil a své souhrnné statistiky.

\begin{enumerate}
    \item Uživatel klikne na tlačítko \todo{\emph{Profil}}.
    \item Aplikace zobrazí obrazovku profilu,
    která zobrazuje informace o profilu a souhrnné statistiky profilu.
\end{enumerate}

\let\thesubsection=\oldsubsection

\subsection{Přehled realizace požadavků}

Přehled realizace případu užití funkčními požadavky lze vidět v
tabulce~\ref{tab:use-case-requirements}.

\begin{table}[h!]
    \centering
    \begin{tabular}{c||c|c|c|c|c|c|c|c|c|c|c} 
        & F1 & F2 & F3 & F4 & F5 & F6 & F7 & F8 & F9 & F10 & F11 \\\hline\hline
        UC1 & X &   &   &   &   &   &   &   &   &   & X \\\hline % login logout
        UC2 &   & X &   &   &   &   &   &   &   &   &   \\\hline % info
        UC3 &   &   &   &   &   &   &   &   &   & X &   \\\hline % settings
        UC4 &   &   & X & X & X & X & X &   &   &   &   \\\hline % game
        UC5 &   &   &   &   &   & X &   &   &   &   &   \\\hline % game exit
        UC6 &   &   &   &   &   &   &   & X & X &   &   \\ % profile
    \end{tabular}
    \caption{Realizace případu užití a splnění požadavků}
    \label{tab:use-case-requirements}
\end{table}
