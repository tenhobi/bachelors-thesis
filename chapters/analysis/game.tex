\section{Popis hry}

Mobilní hra je laděna satiricky do prostředí jaderné elektrárny,
kterou musí uživatelé,
hráči hry v~rozdílných rolích,
zachránit od zničení.
Hra se nazývá \emph{\myAppName}
a~její děj se odehrává v~neznámé jaderné elektrárně.
V~této elektrárně má většina pracovníků dovolenou,
a~proto jsou přítomni jen dva lidé,
technik a~manažer.
Náhlý problém zapříčiní,
že mezi těmito pracovníky byla zablokována cesta,
a~tak se tito dva zaměstnanci musí domluvit pouze hlasem
a~zařídit opravu elektrárny.
Technik vidí všechny řídící prvky,
manažer vidí plány a~manuály.
Jejich cílem je tak společnou spoluprací zabránit neštěstí v~podobě exploze. 

Hra je členěna do několika kapitol
a~každá kapitola obsahuje několik statických misí.
Mise jsou statické v~tom významu,
kdy mají jasně přednastaveny prvky,
které moduly obsahuje,
avšak mění se jejich konkrétní nastavení.
To zajištuje jak možnost znovuopakování,
tak poměřování s~ostatními hráči.

Každá mise,
a~tedy konkrétní hra samotná,
obsahuje několik modulů,
které reprezentují určité řídící prvky elektrárny.
To mohou být nejrůznější tlačítka či přepínače,
ale i~speciální moduly,
reprezentující prvky elektrárny,
které nelze vypnout či opravit a~musí být periodicky kontrolovány,
jako pumpa dodávající chladící vodu do reaktoru.
Kromě modulů je potřeba ve hře okamžitě provádět akce,
které se zobrazí.
Akce jsou většinou jednorázové,
kdy stačí provést danou akci,
ale mohou být i~na dělší dobu,
po kterou musí hráči tuto akci dodržovat.

Konkrétní mise začíná po tom,
co jsou oba pracovníci připraveni
a~je stisknuto tlačítko \emph{AZ-5},
které by mělo zajistit rychlé odstavení reaktoru,
avšak z~neznámého důvodu nefunguje.
Po skončení každé mise,
případně souhrnně za všechny odehrané hry,
jsou zobrazeny statistiky hry.
Statistiky po ukončení hry zobrazí uplynulý čas,
potřebný k~opravě.
Souhrnné statistiky zobrazují
celkový počet odehraných her a~poměr úspěšně splněných her.

Akce jsou přerušením,
které musí hráči okamžitě vykonat.
Příkladem akce může být \emph{zatřeste zařízením}.
Význam jednotlivých akcí je zjevný z~jeho názvu.
Akce jsou využívány zejména v~pokročilejších kapitolách,
jelikož mohou značně komplikovat hru.
