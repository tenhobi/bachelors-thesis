% !TeX root = ../main.tex
\chapter{Konkurenční aplikace}

Po prozkoumání podobného žánru her
– kooperace, komunikace a rychlost reakcí –
dostupných v obchodě Google Play,
byly vybrány konkurenční herní aplikace.
Tyto hry byly vybrány zejména podle popularity a počtu stažení v obchodě
a podle subjektivně vnímaného potenciálu,
který má daná hra k inspiraci pro tvorbu herní aplikace.

V následujících sekcích budou popsány vybrané 4 konkurenční aplikace,
včetně shrnutí kladů a záporů.

\section{DUAL!}

První vybranou hrou je \emph{DUAL!}.
Tato hra využívá zejména prvky kooperace a rychlosti reakcí
a je velmi akčně laděná.
Jak říká první věta stránky hry \cite{seabaa_dual} v obchodě Google Play:
\uv{Mezi lidmi, napříč obrazovkami.}

Hra se rozprostírá přes 2 obrazovky zařízení a herní postava se ovládá natáčením
daného displeje.
Pokud například postava vystřelí na jedné obrazovce,
střela za chvíli doletí i na obrazovku druhou.
Vyvolání příslušných akcí se ovládá určitými kombinacemi dotyků na displej.

UI je velmi jednoduché, avšak přívětivé svými barvami a srozumitelností.
Všechny prvky jsou pochopitelné.

\myFigure{Alpha}{fig:dual}{assets/competitive-apps/dual.jpg}
\FloatBarrier

\subsection{Klady}

TODO text

\subsection{Zápory}

TODO text

\section{Sea Battle 2}

Hra \emph{Sea Battle 2} sice není postavena na kooperaci a komunikaci mezi
hráči,
ale představuje hezkou reprezentaci klasické deskové hry v moderním pojetí
\cite{henrysmithinc_spaceteam}.

Hráči z celého světa mezi sebou na svých zařízeních vedou souboje a vítězstvími
postupně postupují v námořních hodnostech.
Oproti klasické verzi hra disponuje i pokročilým módem,
který obsahuje nejrůznější letadla a pokročilé děla,
které obohatí hru o nové strategie.

UI je lazeno do podoby čtverečkovaného papíru s ručně kreslenými prvky,
což ale nepůsobí moc přívětivě a tvoří několik problémů,
mezi nimiž je například nemožnost přiblížení,
což velmi komplikuje některé úkony.

\myFigure{Alpha}{fig:sea-battle}{assets/competitive-apps/sea-battle.jpg}
\FloatBarrier

\subsection{Klady}

TODO text

\subsection{Zápory}

TODO text

\section{Keep Talking and Nobody Explodes}

Ve hře \emph{Keep Talking and Nobody Explodes} závisí všechno na správné
kooperaci a komunikaci.
Tato velmi oceňovaná hra existuje ve verzi pro desktop, konzole,
virtuální realitu i mobily \cite{steelcrategamesinc_keep}.

Cílem hry je zneškodnění bomby.
Jeden hráč vidí bombu, která je koncipována jako kufřík s několika moduly
a časomírou, která se nebezpečně rychle odpočítává.
Další hráč, případně skupina, mají k dispozici manuál s instrukcemi.
Manuál mohou prohlížet jak v elektronické, tak i ve vytištěné podobě.
Manuál je totiž pro všechny hry shodný.

Veškerá komunikace probíhá pouze slovně.
Hráč s bombou tak nahlásí hráči s manuálem například to,
že vidí červené tlačítko s textem \uv{defuse}.
Druhý hráč pak v manuálu vyhledá daný modul tlačítka, řídí se instrukcemi a
podstatné informace předá prvnímu hráči.

\myFigure{Screenshot hry Keep Talking and Nobody Explodes}{fig:keep-talking}{assets/competitive-apps/keep-talking.jpg}
\FloatBarrier

\subsection*{Klady}

\begin{itemize}
    \item UI je laděné jako pohled na stůl, na kterém je bomba a časomíra,
což dodává atmostféru a vtáhne hráče do děje.
    \item Světla v místnosti blikají, případně se vypínají.
To přidává efekt zejména v situácích,
kdy zbývá už jen pár sekund a bomba není vidět.
    \item Přestože je grafika jednodušší, samotná bomba je velmi přehledná.
\end{itemize}

\subsection*{Zápory}

\begin{itemize}
    \item U složitějších modulů je manuál občas hůře pochopitelný.
\end{itemize}

\section{Spaceteam}

Poslední hra, \emph{Spaceteam}, stojí na rychlosti reakcí a velmi rychlé
komunikaci, což shrnuje první věta na stránce hry \cite{henrysmithinc_spaceteam}
v obchodě Google Play: \uv{Mačkáš rád tlačítka a řveš na své kamarády?}

Herní plocha je plná náhodných tlačítek, přepínačů, posouvátek a dalších věcí,
se kterými musí hráči pracovat podle přicházejících příkazů.

Zapeklitost komunikace spočívá v tom,
že příkazy se netýkají pouze prvků na nadaném zařízení,
ale i prvků ze zařízení ostatních lidí.
Některé příkazy také cílí na všechny hráče.
Příkladem takévého příkazu je například zatřesení všemi zařízeními najednou.

\myFigure{Screenshot hry Spaceteam}{fig:spaceteam}{assets/competitive-apps/spaceteam.jpg}
\FloatBarrier

\subsection*{Klady}

\begin{itemize}
    \item UI úvodní obrazovky a přihlašování do hry vypadá velmi efektně.
    \item Tlačítka i příkazy obsahují řadu technologických pojmů,
což dodává hře na správné atmosféře.
\end{itemize}

\subsection*{Zápory}

\begin{itemize}
    \item Herní obrazovka má poněkud zastaralé, až nevkusné, prvky.
    \item Hra probíhá až příliš rychle.
Nutno přiznat, že tento fakt není nutně nevýhoda,
avšak ubírá to na prvku kvalitnější komunikace.
    \item Hra nemá možnost přihlášení,
a tedy shromažďování a uchovávání statistik.
\end{itemize}
