\chapter{Konkurenční aplikace}

Po prozkoumání podobného žánru her – kooperace, komunikace, logika a rychlost reakcí – dostupných v obchodu Google Play, byly vybrány 4 konkurenční herní aplikace.

Dobrých her splňující kritéria bylo v daném obchodě samozřejmě k nalezení vícero. Vybrány byly ty nejzajímavější hry, ze kterých se může \emph{\appName{}} nejvíce inspirovat.

\section{DUAL!}

První vybranou hrou je \emph{DUAL!}. Hra využívá zejména prvky kooperace a rychlosti reakcí a je velmi akčně laděná. Jak říká první věta stránky hry \cite{seabaa_dual} v obchodě Google Play: \uv{Mezi lidmi, napříč obrazovkami.}

Hra se rozprostírá přes 2 obrazovky zařízení a herní postava se ovládá natáčením daného displeje. Pokud například postava vystřelí na jedné obrazovce, střela za chvíli doletí i na obrazovku druhou. Vyvolání příslušných akcí se ovládá určitými kombinacemi dotyků na displej.

UI je velmi jednoduché, avšak přívětivé svými barvami a srozumitelností. Všechny prvky jsou pochopitelné.

\section{Sea Battle 2}

Hra \emph{Sea Battle 2} sice není postavena na kooperaci a komunikaci mezi hráči, ale představuje hezkou reprezentaci klasické deskové hry v moderním pojetí \cite{henrysmithinc_spaceteam}.

Hráči z celého světa mezi sebou na svých zařízeních vedou souboje a vítězstvími postupně postupují v námořních hodnostech. Oproti klasické verzi hra disponuje i pokročilým módem, který obsahuje nejrůznější letadla a pokročilé děla, které obohatí hru o nové strategie.

UI je lazeno do podoby čtverečkovaného papíru s ručně kreslenými prvky, což ale nepůsobí moc přívětivě a tvoří několik problémů, mezi nimiž je například nemožnost přiblížení, což velmi komplikuje některé úkony. 

\section{Keep Talking and Nobody Explodes}

Ve hře \emph{Keep Talking and Nobody Explodes} závisí všechno na správné kooperaci, komunikaci a rychlosti. Tato velmi oceňovaná hra existuje ve verzi pro desktop, konzole, virtuální realitu i mobily \cite{steelcrategamesinc_keep}.

Celá hra stojí okolo problému zneškodnění bomby. Jeden hráč vidí bombu, která je koncipována jako kufřík s několika moduly a časomírou, která se nebezpečně rychle odpočítává. Další hráč, případně skupina, mají k dispozici manuál s instrukcemi. Manuál mohou prohlížet jak v elektronické, tak i ve vytištěné podobě. Veškerá komunikace probíhá pouze slovně.

Hráč s bombou tak nahlásí hráči s manuálem například to, že vidí červené tlačítko s textem \uv{defuse}. Druhý hráč pak v manuálu vyhledá daný modul tlačítka, řídí se instrukcemi a podstatné informace předá prvnímu hráči.

UI je laděné jako pohled na stůl, na kterém je bomba a časomíra, což dodává atmostféru a vtáhne do děje. Světla v místnosti navíc blikají, případně se vypínají, což přidává efekt zejména v situácích, kdy zbývá už jen pár sekund a bomba není vidět. Přestože je grafika jednodušší, samotná bomba je velmi přehledná.

\section{Spaceteam}

Poslední hra \emph{Spaceteam} stojí na rychlosti reakcí a velmi rychlé komunikaci, což shrnuje věta na stránce hry \cite{henrysmithinc_spaceteam} v obchodě Google Play: \uv{Mačkáš rád tlačítka a řveš na své kamarády?}

Herní plocha je plná náhodných tlačítek, přepínačů, posouvátek a dalších věcí, se kterými musí hráči pracovat podle přicházejících \uv{technologicko-nemyslných} příkazů. Zapeklitost komunikace spočívá v tom, že příkazy se netýkají pouze daného zařízení, ale i zařízeních dalších lidí.

UI úvodní obrazovky a přihlašování vypadá efektně, avšak samotná herní obrazovka má poněkud zastaralé prvky. Nutno uznat, že v tomto konkrétním případě to nic nemění na zážitku ze hry.

WTF Je toto? lol DOPÍČI kunda píča rofl
