\section{State management}

Používání proměnných či funkcí pro řízení vykreslování aplikace může být
snadné a logické řešení pro menší aplikace,
avšak pro větší aplikace
--- a zejména aplikace, které dbají na udržitelnost a rozšiřitelnost ---
je vhodné využít některé z řešení pro správu stavů dílčích částí aplikace,
známé především jako \uv{state management}.
 
Stavy nám umožňují přemýšlet o UI ve Flutteru deklarativně.
Ve Flutteru platí,
že UI je výsledkem \mintinline{dart}|build| metody,
která pracuje s aktuálním stavem aplikace.
Pro změnu barvy widgetu se tak nevolá funkce jako
\mintinline{dart}|widget.setColor|,
ale změní se stav a podle něj se příslušně překreslí UI.
\cite{flutter_state_mgmt_declarative}

Pojem \uv{stav} můžeme chápat vícero způsoby.
V této práci je stav chápán jako všechna data,
která jsou potřeba k překreslení UI v daný moment.
Dle \cite{flutter_state_mgmt_ephemeral_vs_app} lze stavy dělit do dvou typů:

\begin{description}
    \item[Ephemeral state] Někdy také nazývaný \uv{UI state} nebo
\uv{local state},
je typ stavu využívaný většinou pro dílčí část UI
a ostatní widgety jen zřídka potřebují přístup k těmto stavům.
Příkladem je aktivní stránka v \mintinline{dart}|PageView|,
index zvolené záložky \mintinline{dart}|BottomNavigationBar|,
aktuální stav animace atp.
Tento typ je většinou řešen jako \mintinline{dart}|State| třída pro daný
\mintinline{dart}|StatefulWidget| pomocí vlastností třídy a metody
\mintinline{dart}|setState()| pro aktualizaci stavu.
    \item[App state] Někdy také nazývaný \uv{shared state} je typ stavu
sdílený mezi několika částmi aplikace.
Příkladem je uživatelské nastavení, notifikace, nákupní košík,
informace o přihlášení atp.
\end{description}

Neexistuje přesné pravidlo,
jak určit typ stavu,
a spolu s tím jak implementovat daný stav v aplikaci.
Všechny stavy mohou být implementovány pomocí \mintinline{dart}|State| a
\mintinline{dart}|setState()|,
a pro mnoho malých aplikací to může být nejjednodušší řešení.
Stejně tak lze všechny stavy implementovat jako \emph{App state}.

Následující sekce obsahují popis možností pro práci s \emph{App state}.

\subsection{Předávání dat a callbacků}

Ve Flutteru je samozřejmě možné mezi widgety předávat data konstruktory
a zpracovávat reakce pomocí callbacků.
Pomocí callbacků lze interagovat se stavem,
který se přesune do nejvýše postaveného widgetu ve stromu widgetů.
Důvod je ten,
že ve Flutteru
--- který je deklarativní ---
je nutné pro překreslení UI spustit \mintinline{dart}|build|.
To je zajisté funkční a snadné řešení.
Avšak mnohokrát je nutné data a callbacky předávat přes více widgetů,
čímž se aplikace velmi rychle stane nelehce rozšiřitelnou.
\cite{flutter_state_mgmt_simple}

Jak lze vidět v ukázce \ref{code:state-callback},
vnořenému widgetu jsou předány data
a pro změnu stavu z vnořeného widgetu je tomuto widgetu předán callback.
Ten se využije pro interakce se stavem.

\begin{listing}
    \caption{Manipulace se stavem pomocí předávání dat a callbacku
\cite{flutter_state_mgmt_simple}}
    \label{code:state-callback}
    \begin{minted}{dart}
    @override
    Widget build(BuildContext context) {
        return SomeWidget(
            MyListItem(myData, myTapCallback),
        );
    }

    void myTapCallback(Item item) {
        print('uživatel klepnul na $item');
    }
    \end{minted}
\end{listing}

Flutter má naštěstí způsoby,
jak mohou widgety sdílet stavy bez toho,
aniž by se snižovala rozšiřitelnost aplikace pomocí nepřehledného předávání
dat a callbacků.
Stejně jako vše ve Flutteru,
i toto je řešeno pomocí widgetů.
Za zmíňku stojí \mintinline{dart}|InheritedWidget|,
na který lze jednoduše odkázat odkudkoli z jeho podstromu pomocí
\mintinline{dart}|MyInheritedWidget.of(context)|.
Tento widget a další funkcionality zaobaluje knihovna
\mintinline{dart}|provider|,
která zjednodušuje použití a snižuje množství potřebného kódu.

\subsection{MobX}

MobX je knihovna pro state management, která 

\todo{Napsat o MobX.}
\blind[2]

\subsection{Redux}

\todo{Napsat o Reduxu.}
\blind[2]

\subsection{BLoC}

\todo{Napsat o BLoCu obecně.}
\blind[2]

%https://github.com/brianegan/flutter_architecture_samples/tree/master/bloc_library

\subsubsection{Implementace Felixe Angelova}

\todo{Napsat v čem je fajn implementace od Felixe.}
\blind[3]

\subsection{Zhodnocení}

\todo{Napsat proč si vybírám BLoC a nějaké porovnání.}
\blind[2]

