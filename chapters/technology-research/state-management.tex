\section{Správa stavů}

Používání proměnných či funkcí pro řízení vykreslování aplikace může být
snadné a~logické řešení pro menší aplikace,
avšak pro větší aplikace
--- a~zejména aplikace, které dbají na udržitelnost a~rozšiřitelnost ---
je vhodné využít některé z~řešení pro správu stavů dílčích částí aplikace,
známé především jako \emph{state management}.

Stavy nám umožňují přemýšlet o~UI ve Flutteru deklarativně.
Ve Flutteru platí,
že UI je výsledkem \mintinline{dart}|build| metody,
která pracuje s~aktuálním stavem aplikace.
Pro změnu barvy widgetu se tak nevolá funkce jako
\mintinline{dart}|widget.setColor|,
ale změní se stav a~podle něj se příslušně
překreslí UI.~\cite{flutter_state_mgmt_declarative}

Pojem \emph{stav} můžeme chápat více způsoby.
V~této práci je stav chápán jako všechna data,
která jsou potřeba k~překreslení UI
v~daný moment.
Dle~\cite{flutter_state_mgmt_ephemeral_vs_app} lze stavy dělit do dvou typů:

\begin{description}
    \item[Ephemeral state] Někdy také nazývaný \emph{UI state} nebo
    \emph{local state},
    je typ stavu využívaný většinou pro dílčí část UI
    a~ostatní widgety jen zřídka potřebují přístup k~těmto stavům.
    Příkladem je aktivní stránka ve widgetu \mintinline{dart}|PageView|,
    index zvolené záložky \mintinline{dart}|BottomNavigationBar|,
    aktuální stav animace a~podobně.
    Tento typ je většinou řešen jako třída \mintinline{dart}|State| pro~daný
    \mintinline{dart}|StatefulWidget| pomocí vlastností třídy a~metody
    \mintinline{dart}|setState()| pro aktualizaci stavu.
    \item[App state] Někdy také nazývaný \emph{shared state} je typ stavu
    sdílený mezi několika částmi aplikace.
    Příkladem je uživatelské nastavení, notifikace, nákupní košík,
    informace o~přihlášení a~podobně.
\end{description}

Neexistuje přesné pravidlo,
jak určit typ stavu a~spolu s~tím jak implementovat daný stav v~aplikaci.
Všechny stavy mohou být implementovány pomocí \mintinline{dart}|State|
a~\mintinline{dart}|setState()|,
a~pro mnoho malých aplikací to může být nejjednodušší řešení.
Stejně tak lze všechny stavy implementovat jako
App state.~\cite{flutter_state_mgmt_ephemeral_vs_app}

Následující sekce obsahují popis možností pro práci s~App state.
Každá sekce uvede danou možnost a~popíše klíčové vlastnosti.
Na závěr je provedeno porovnání a~zhodnocení.

\subsection{Předávání dat a~callbacků}
\label{sec:data-callback-transfer}

Ve Flutteru je samozřejmě možné mezi widgety předávat data konstruktory
a~zpracovávat reakce pomocí callbacků. 
Pomocí callbacků lze interagovat se stavem,
který se přesune do nejvýše postaveného widgetu ve stromu widgetů.
Důvod je ten,
že ve Flutteru,
který je deklarativní,
je nutné pro překreslení UI spustit
\mintinline{dart}|build|~\cite{flutter_state_mgmt_simple}.
To je zajisté funkční a~snadné řešení,
avšak mnohokrát je nutné data a~callbacky předávat přes více widgetů,
čímž se aplikace velmi rychle stane nelehce rozšiřitelnou.

Jak lze vidět v~ukázce~\ref{code:state-callback},
vnořenému widgetu jsou předány data
a~pro změnu stavu z~vnořeného widgetu je tomuto widgetu předán callback.
Ten se využije pro interakce se stavem.

\begin{listing}
    \caption{Manipulace se stavem pomocí předávání dat a~callbacku
\cite{flutter_state_mgmt_simple}}
    \label{code:state-callback}
    \begin{minted}{dart}
@override
Widget build(BuildContext context) {
    return SomeWidget(
        MyListItem(myData, myTapCallback),
    );
}

void myTapCallback(Item item) {
    print('user tapped on $item');
}
    \end{minted}
\end{listing}

Flutter má způsoby,
jak mohou widgety sdílet stavy bez toho,
aniž by se snižovala rozšiřitelnost aplikace pomocí nepřehledného předávání
dat a~callbacků.
Stejně jako vše ve Flutteru,
i~toto je řešeno pomocí widgetů.
Za zmínku stojí \mintinline{dart}|InheritedWidget|,
na který lze jednoduše odkázat odkudkoli z~jeho \mbox{podstromu} pomocí
\mintinline{dart}|MyInheritedWidget.of(context)|.
Tento widget a~další funkcionality zaobaluje knihovna Provider,
která zjednodušuje použití a~snižuje množství potřebného
kódu.~\cite{flutter_state_mgmt_simple}

\pagebreak
\subsection{MobX}

MobX je knihovna pro správu stavů,
která implementuje vzor observer.
Díky jednomuchému rozhraní je vhodná pro začátečníky
a~je používána i~ve velkých projektech.
Dle~\cite{mobx_core_concepts} pracuje knihovna se třemi základními koncepty:

\begin{description}
    \item[Observables] Reprezentují stavy v~podobě libovolného objektu s~daty.
    Stavy mohou být i~odvozené na základě jiných odvozených a~základních stavů.
    Každá změna stavu vyvolá upozornění pozorovatelů, a~tedy jejich reakce.
    \item[Actions] Popisují změny stavů.
    Stavy se nemění přímo,
    ale pomocí akcí,
    které změnám přidávají sémantický význam.
    Namísto přímého volání \mintinline{dart}|value++| se volá akce
    \mintinline{dart}|increment()|.
    Akce vyvolává změnu stavu.
    \item[Reactions] Automaticky sledují změnu stavů.
    Neprodleně po každé změně se provede reakce.
    Reakce vyvolávají akce.
\end{description}

\begin{listing}
    \caption{Ukázka kódu počítadla v~knihovně MobX~\cite{mobx_core_concepts}}
    \label{code:state-mobx}
    \begin{minted}{dart}
import 'package:mobx/mobx.dart';

part 'counter.g.dart';

class Counter = CounterBase with _$Counter;

abstract class CounterBase with Store {
    @observable
    int value = 0;

    @action
    void increment() {
        value++;
    }
}
    \end{minted}
\end{listing}

Z~výpisu kódu~\ref{code:state-mobx} lze vypozorovat,
že knihovna pracuje s~generovanými kódy z~pomocných generátorů.
Tím se dle~\cite{mobx_core_concepts} snaží redukovat množství kódu,
které by bylo jinak vyžadováno napsat k~docílení stejné funkčnosti.
Knihovna také využívá anotace,
kterými je kód zpřehledněn.
Generování kódu však není nejrychlejší,
což je jedna z~nevýhod tohoto přístupu.

\pagebreak
\subsection{Redux}

Jedna z~nejznámějších knihoven řešící správu stavů je knihovna Redux.
Dle~\cite{redux_basics} pracuje Redux se třemi základními koncepty:

\begin{description}
    \item[Actions] Obsahují data,
    která se posílají z~aplikace do \emph{Store}
    a~jsou jediným zdrojem informací.
    \item[Reducers] Popisují, jak se změní stav aplikace v~závislosti na akci.
    \item[Store] Pojí předešlé dva koncepty dohromady.
    Drží stav aplikace,
    určuje počáteční stav
    a~poskytuje aktuální stav.
\end{description}

\begin{listing}
    \caption{Ukázka kódu počítadla v~knihovně Redux~\cite{redux_basics}}
    \label{code:state-redux}
    \begin{minted}{dart}
import 'package:redux/redux.dart';

enum CounterActions {
    increment
}

class AppState {
    int value;

    AppState({ 
        this.value = 0,
    });
}

AppState counterReducer(AppState state, action) {
    switch (action) {
        case CounterActions.increment:
            return AppState(
                value: state.value++,
            );
        default:
            return state;
    }
}
    \end{minted}
\end{listing}

Dle~\cite{redux_basics} knihovna Redux obsahuje právě jeden \emph{Store}
a~členění se provádí pomocí kompozice \emph{Reducers}.
Z~výpisu kódu~\ref{code:state-redux} je vidět,
že ačkoli je zápis přehledný,
ve větší aplikaci nemusí být omezení pouze jednoho \emph{Store} vhodné.

\subsection{BLoC}

Oproti předchozím příkladům není Business Logic Component (dále jen BLoC)
knihovna,
ale návrhový vzor.
Tento vzor je tvořen několika jednoduchými pravidly,
kterými se každá implementace musí řídit,
přičemž konkrétní prvky implementace jsou přenechány na
vývojáře.
Pojmem BLoC také označujeme třídu,
která uchovává stav dodržuje daný vzor~\cite{googledevelopers_bloc}. 
Dle~\cite{googledevelopers_bloc} těmito pravidly pro návrh jsou:

\begin{enumerate}
    \item Inputs and outputs are simple
    \mintinline{dart}|Stream|/\mintinline{dart}|Sink| only.
    \item Dependencies must be injectable and platform agnostic.
    \item No platform branching allowed.
    \item Implementation can be whatever you want
    if you follow the previous rules.
\end{enumerate}

Spolu s~těmito pravidly pro vzor BLoC byla představena
pravidla i~pro návrh UI.
Ta se zabývají zejména vztahem mezi komponentou UI, komponentou BLoC
a~podobou dat,
které se posílají do a~z~každé třídy BLoC~\cite{googledevelopers_bloc}.
Dle~\cite{googledevelopers_bloc} těmito pravidly jsou:

\begin{enumerate}
    \item Each \auv{complex enough} component has a~corresponding BLoC.
    \item Components should send inputs \auv{as is}.
    \item Components should show outputs as close as possible to \auv{as is}.
    \item All branching should be based on simple BLoC boolean outputs.
\end{enumerate}

Tento návrhový vzor vznikl za účelem oddělení kódu s~logikou
od závislostí na konkrétní platformě~\cite{googledevelopers_bloc}.
Oddělením vrstvy UI a~logiky má vzor za cíl tvořit kód více udržitelný
a~testovatelný~\cite{flutterando_analyzing_bloc_mobx}.
Oproti knihovně Redux si lze z~uvedených pravidel všimnout,
že už samotný návrhový vzor BLoC pracuje s~více zdroji pro uchování stavu.

\subsubsection{Knihovna Bloc}

Jednou z~implementací tohoto vzoru je knihovna Bloc od Felixe
Angelova.
Tato implementace se zaměřuje na potřeby vývojářů,
například zpřístupnění informací o stavech aplikace v~daný moment,
snadné testování, znovupoužitelnost, rychlý vývoj a~mnoho dalšího~\cite{bloclibrary_whybloc}. 
Dle zdroje~\cite{bloclibrary_coreconcepts} pracuje knihovna Bloc se čtyřmi
základními koncepty: 

\begin{description}
    \item[Events] Reprezentují události v~podobě libovolného objektu
    a~jsou jediným vstupem do \emph{Bloc}.
    Tyto objekty jsou většinou vytvářeny dle potřeb UI.
    \item[States] Představují stavy,
    jednotlivých částí aplikace,
    taktéž v~podobě libovolného objektu.
    S~těmito objekty následně pracuje UI.
    \item[Transitions] Popisují změnu z~jednoho stavu do jiného.
    Transakce se skládají z~aktuálního stavu, události a~následujícího stavu.
    Sekvence těchto záznamů tedy tvoří historii změn stavů.
    \item[Blocs] Reprezentují třídy,
    které pojí předešlé koncepty dohromady.
    Tyto třídy přijímají události, produkují stavy a~jednotlivé změny
    zaznamenávají v~podobě transakcí.
    Jednotlivé změny stavů jsou prováděny v~metodě
    \mintinline{dart}|mapEventToState(event)|,
    která produkuje tok stavů.
\end{description}

\begin{listing}
    \caption{Ukázka kódu počítadla v~knihovně
    Bloc~\cite{bloclibrary_coreconcepts}}
    \label{code:state-bloc}
    \begin{minted}{dart}
import 'package:bloc/bloc.dart';

enum CounterEvent { increment }

class CounterBloc extends Bloc<CounterEvent, int> {
    @override
    int get initialState => 0;

    @override
    Stream<int> mapEventToState(CounterEvent event) async* {
        switch (event) {
            case CounterEvent.increment:
                yield state + 1;
                break;
        }
    }
}
    \end{minted}
\end{listing}

Z~výpisu kódu~\ref{code:state-bloc} lze vidět,
že knihona Bloc pracuje asynchronně pomocí toků dat za využití
tříd \mintinline{dart}|Stream| a~\mintinline{dart}|Sink|.
To umožňuje pro jednu událost snadno vyvolat několik změn
a~zároveň reagovat na změny stavu asynchronně i~v~UI.

\subsection{Zhodnocení}

Vyzdvihování stavu,
jak je popsáno v~sekci~\ref{sec:data-callback-transfer},
není ideální řešení pro udržitelnost a~rozšiřitelnost.
Knihovna Provider sice řeší vkládání závislostí,
avšak neřeší samotnou správu stavů.

Knihovna MobX silně závisí na generování kódu,
což může mít vliv na~její rychlost.
Oproti tomu jsou knihovny Bloc a~Redux sice náročnější na~porozumění,
avšak nabízí asynchronní přístup,
který je vhodný pro práci s~daty ve velkých aplikacích.

Při porovnání knihoven Redux a~Bloc lze vidět,
že se obě knihovny zdánlivě velmi podobají.
Konkrétní implementace i~principy se však rozcházejí.
Jeden z~největších viditelných rozdílů je počet zdrojů,
které udržují stav.
Pro~\mbox{knihovnu} Redux existuje pouze jeden zdroj, \emph{Store}.
V~knihovně Bloc se \mbox{používá} libovolný počet těchto zdrojů,
reprezentovaných třídami \mintinline{dart}|Bloc|.
\mbox{Zároveň} s~tím se v~knihovně Redux mění pouze části stavu,
kdežto v~knihovně Bloc se při změně nahrazuje celý stav stavem novým.
Knihovna Bloc také svou implementací umožňuje velmi snadné asynchronní
změny stavu,
čímž umožňuje již zmiňované vícenásobné vyvolání změn pro jednu událost.

V~této práci bude tedy z~výše uvedených důvodů využívána knihovna Bloc.
Tato knihovna umožňuje nejsnadnější asynchronní práci,
sledování změn stavů pomocí transakcí
a~velmi dobře podporuje oddělení od externích závislostí.
