\section{Senzory}

Použití senzorů v herních aplikacích je velmi časté.
Senzory totiž mohou dodat aplikaci data,
která jí pomohou obohatit o detaily,
případně na nich může aplikace postavit funkčnost dílčích částí.
Počet ušlých kroků,
navigace pomocí kompasu,
ovládání závodního auta natáčením telefonu,
nebo počítání provedených kliků
--- všechny tyto příklady využívají senzory zařízení.
\cite{sensors} 

Mobilní zařízení má několik vestavěných senzorů,
které měří pohyb, orientaci v prostoru a plno dalších věcí.
Každé zařízení může mít rozdílný počet senzorů
a senzory mohou být buď hardwarové, nebo softwarové.
Softwarové senzory zpracovávají data z jednoho nebo více hardwarových senzorů
a imitují očekávaný výstup.
Dle zdroje~\cite{sensors} se senzory dělí na tři základní typy:

\begin{description}
    \item[Senzory pohybu] Snímají otřesy a posuny,
    tedy zrychlení a rotaci,
    pomocí tří os.
    Příkladem je akcelerometr, senzor gravitace, gyroskop
    a rotační vektorový senzor.
    \item[Senzory polohy] Snímají fyzickou pozici zařízení.
    Případem je magnetometr a senzor přiblížení.
    \item[Senzory prostředí] Snímají hodnoty prostředí
    jako je teplota, tlak či vlhkost vzduchu.
    Příkladem je termometr a barometr.
\end{description}

Pro herní aplikace jsou nejdůležitější senzory pohybu a senzory polohy.
Senzory prostředí závisí na datech prostředí,
které se nemění tak často,
a tedy nemají pro hru větší využití.
Množství využítí naopak umožňují senzory pohybu a polohy.
\cite{sensors_android}
V násedujících sekcích proto budou popsány senzory akcelerometr, gyroskop
a senzor přiblížení.

\subsection{Akcelerometr}

Akcelerometr určuje akceleraci v rámci os daného mobilního zařízení
spolu s gravitací,
která na toto mobilní zařízení v jednotlivých jeho osách
působí~\cite{sensors_motion}.
Z tohoto důvodu akcelerometr naměří výsledky i v případech,
kdy mobilní zařízení například leží na stole,
konkrétně s hodnotou udanou
gravitační silou $g = -9,81 m/s^2$~\cite{sensors_motion}.
Tento senzor je tedy vhodný pro detekování pohybu zařízení
a lze říci, že jej obsahují všechna zařízení se systémy Android a iOS.

\subsection{Gyroskop}

Gyroskop měří pohyb rotace jednotlivých os mobilního
zařízení~\cite{sensors_motion}.
Oproti akcelerometru tento senzor při nehybném zařízení
nevydává žádná data,
senzor totiž měří pouze změnu.
Změna rotace je udávaná jako radiány za sekundu~\cite{sensors_motion}.
Tento senzor lze využít například pro rotaci ve 3D při 360° pohledu,
kdy uživatel sedí na otočné židli a otáčením židle prohlíží okolí v aplikaci.

\subsection{Senzor přiblížení}

Senzor přiblížení určuje vzdálenost objektu od zařízení.
Senzor je běžně umístěn na přední straně mobilního zařízení
a určuje absolutní vzdálenost,
nicméně některé druhy tohoto senzoru určují pouze
stavy \emph{blízko} a \emph{daleko}.~\cite{sensors_position}
Například systém Android vypíná obrazovku,
zatímco uživatel telefonuje se zařízením přiloženým u ucha.

Tento senzor lze využít například pro snímání vzdálenosti ruky od zařízení,
nebo v kombinaci s gyroskopem k určení stavu,
kdy je mobil položený na stole a otočený směrem k němu.

\subsection{Zhodnocení}

Jak již bylo uvedeno,
nejvyšší uplatnění v herních aplikacích mají senzory pohybu a polohy.
V předešlých kapitolách byly popsány senzory,
které lze běžně v herních aplikacích využít.
V praktické části práce proto budou moci být použity právě tyto senzory,
které nabízí vhodné rozhraní,
z volně dostupných balíčků.
