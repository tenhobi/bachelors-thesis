\section{Architektura}

\todo{Popsat obecně architekturu, k čemu se používá, rozdíl oproti patternu?,
...}
\blind[5]

%TOHLE !! https://crosp.net/blog/software-architecture/clean-architecture-part-1-databse-vs-domain/
%TOHLE !!! https://github.com/igorwojda/Android-Showcase

\subsection{Moduly}

\todo{Možnosti jak pracovat s moduly/features pro UDRŽITELNÝ vývoj}
\blind[3]

%https://martinfowler.com/bliki/PresentationDomainDataLayering.html

\subsection{Tradiční n-vrstvá architektura}

\todo{Jak funguje tradiční n-vrstvá architektura.}
\blind[3]

%https://jeffreypalermo.com/2008/07/the-onion-architecture-part-1/

%https://jeffreypalermo.com/2008/08/the-onion-architecture-part-3/

%http://www.pinte.ro/Blog/DesignPatterns/Clean-Architecture-An-alternative-to-traditional-three-layer-database-centric-applications/37

%https://docs.microsoft.com/cs-cz/dotnet/architecture/modern-web-apps-azure/common-web-application-architectures

%https://jeremiahflaga.github.io/2019/05/20/vertical-slice-architecture-is-it-incompatible-with-clean-architecture/

% Asi jen Tradiční n-vrstvá vs Clean Architecture

%https://softwareengineering.stackexchange.com/questions/371966/is-clean-architecture-by-bob-martin-a-rule-of-thumb-for-all-architectures-or-i

%https://khalilstemmler.com/articles/software-design-architecture/domain-driven-design-vs-clean-architecture/

%https://marcduerst.com/2019/09/22/chapter-2-clean-architecture-domain-driven-design/

\subsection{Clean Architecture}

\todo{\emph{\uv{Good architecture makes the system easy to understand, easy to develop, easy to maintain, and easy to deploy. The ultimate goal is to minimize the lifetime cost of the system and to maximize programmer productivity. ― Robert C. Martin, Clean Architecture}}}

\todo{\emph{\uv{Good software systems begin with clean code. On the one hand, if the bricks aren’t well made, the architecture of the building doesn’t matter much. On the other hand, you can make a substantial mess with well-made bricks. This is where the SOLID principles come in. ― Robert C. Martin, Clean Architecture}}}

\todo{\emph{\uv{Software has two types of value: the value of its behavior and the value of its structure. The second of these is the greater of the two because it is this value that makes software soft. ― Robert C. Martin, Clean Architecture}}}

\todo{\emph{\uv{The way you keep software soft is to leave as many options open as possible, for as long as possible. What are the options that we need to leave open? They are the details that don’t matter. ― Robert C. Martin, Clean Architecture}}}

%https://gist.github.com/ygrenzinger/14812a56b9221c9feca0b3621518635b

\todo{Co je Clean Architecture \cite{martin_2018_clean}, proč je to fajn používat atp.}
\blind[4]

%https://hackernoon.com/hammering-at-clean-architecture-1wbr3cgo

%https://github.com/ResoCoder/flutter-tdd-clean-architecture-course

%https://stackoverflow.com/questions/23479879/clean-architecture-vs-onion-architecture

%TOHLE !! https://crosp.net/blog/software-architecture/clean-architecture-part-2-the-clean-architecture/

%https://blog.cleancoder.com/uncle-bob/2012/08/13/the-clean-architecture.html

%https://github.com/android10/Android-CleanArchitecture

%https://dev.to/bosepchuk/why-i-cant-recommend-clean-architecture-by-robert-c-martin-ofd

%https://proandroiddev.com/clean-architecture-data-flow-dependency-rule-615ffdd79e29

%https://www.freecodecamp.org/news/a-quick-introduction-to-clean-architecture-990c014448d2/

%https://proandroiddev.com/multiple-ways-of-defining-clean-architecture-layers-bbb70afa5d4a

%https://github.com/bufferapp/android-clean-architecture-boilerplate

% https://github.com/bufferapp/android-clean-architecture-boilerplate

\subsection{Zhodnocení}

\todo{Popsat proč si vybírám Clean Architecture.}
\blind[2]
