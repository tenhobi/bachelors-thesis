\section{Testování}

Testování se v průběhu let stalo nedílnou součástí vývoje aplikací.
Je vhodné pro všechny velikostí aplikací,
především proto,
že testování vede k vyšší kvalitě kódu
a tím k snadnější udržitelnosti aplikace~\cite{testing_quality}.
Cílem testů je ověřit a dále kontrolovat,
že specifické části aplikace fungují
a že zachovávají tuto funkčnost při jejich úpravě~\cite{testing_quality}.
\emph{\uv{Dijkstra once said,
\buv{Testing shows the presence, not the absence, of bugs.}
In other words, a program can be proven incorrect by a test,
but it cannot be proven correct.
All that tests can do, after sufficient testing effort,
is allow us to deem a program to be correct enough for our
purposes.}}~\cite[kapitola~4]{martin_clean_architecture}

V této podkapitole jsou popsány typy testů,
které pomáhají k snažšímu testování.
Také jsou zmíněny metody a přístupy,
které umožní tvoření kvalitnějších testů
a využití testů při připojování kódu ve verzovacích systémech.

\subsection{Typy testů}

Testovat je možné i ručně.
To však je však z globálního hlediska neefektivní a zdlouhavé.
V praxi se proto využívají automatizované testy,
které lze spustit samostatně kdykoli během cyklu vývoje
aplikace~\cite{testing_flutter}.
Dle~\cite{testing_flutter} se automatizované testy dělí do těchto kategorií:

\begin{description}
    \item[Unit testy] Testují jednu funkci, metodu nebo třídu.
    Tento typ testů je snadné vytvořit,
    rychle se spouští
    a jejich smyslem je ověřit správnou funkčnost dané části.
    Externí závislosti jsou nahrazeny jejich testovací kopií,
    která neprovádí žádnou funkčnost.
    \item[Widget testy] Někdy také nazývané jako \emph{component testy},
    testují jeden widget jako celek.
    Jejich smyslem je ověřit,
    jestli se daný widget chová a vypadá dle očekávání.
    Tento typ testů je složitější vytvořit a udržovat.
    \item[Integrační testy] Testují celou aplikaci nebo její části.
    Tento typ testů je složité vytvořit,
    mají velké množství závislostí a jsou velmi pomalé na spuštění.
    Na druhou stranu testují korektnost aplikace,
    či její části,
    jako celku.
\end{description}

Ve výpisu kódu~\ref{code:test-unit} lze vidět příklad unit testu.
Unit testování vyžaduje jako závislost pro vývoj balíček
\mintinline{dart}|test|.
Příklad obsahuje čítač reprezentovaný třídou \mintinline{dart}|Counter|,
která umí svou hodnotu \mintinline{dart}|value| inkrementovat pomocí
metody \mintinline{dart}|increment()|.
Jak lze vidět,
použitý balíček umožní vývojáři využívat například metody
\mintinline{dart}|test()| a \mintinline{dart}|expect()|,
díky kterým lze popsat očekávaný stav každého testu.~\cite{testing_flutter_unit}

Ve výpisu kódu~\ref{code:test-widget} lze naopak vidět příklad widget testu.
Testování widgetů vyžaduje jako závislost pro vývoj balíček
\mintinline{dart}|flutter_test|.
Příklad ukazuje ukázku toho,
jak lze u testovaného widgetu otestovat očekávané položky \emph{title}
a \emph{message}.
Princip funkčnosti je stejný,
jako u balíčku \mintinline{dart}|test|,
jen se namísto metody \mintinline{dart}|test()|
využívá metoda \mintinline{dart}|testWidgets()|.~\cite{testing_flutter_widget}

\begin{listing}
    \caption{Ukázka unit testu~\cite{testing_flutter_unit}}
    \label{code:test-unit}
    \begin{minted}{dart}
import 'package:test/test.dart';
import 'package:counter_app/counter.dart';

void main() {
    test('Counter value should be incremented', () {
    final counter = Counter();

    counter.increment();

    expect(counter.value, 1);
    });
}
    \end{minted}
\end{listing}

\begin{listing}
    \caption{Ukázka widget testu~\cite{testing_flutter_widget}}
    \label{code:test-widget}
    \begin{minted}{dart}
import 'package:flutter/material.dart';
import 'package:flutter_test/flutter_test.dart';

void main() {
    testWidgets(
        'MyWidget has a title and message',
        (WidgetTester tester) async {
        await tester.pumpWidget(MyWidget(title: 'T', message: 'M'));

        final titleFinder = find.text('T');
        final messageFinder = find.text('M');

        expect(titleFinder, findsOneWidget);
        expect(messageFinder, findsOneWidget);
        },
    );
}
    \end{minted}
\end{listing}

\subsection{Code coverage}

Code coverage znamená pokrytí kódu.
V testování to je pojem,
který značí,
do jaké míry jsou třídy a jejich vlastnosti a metody otestovány.
Určitě ale neplatí,
že 100 \% code coverage značí,
že je aplikace správně otestována.
Tato metoda identifikuje kód,
který již není používán,
nebo ani nikdy nebyl,
nebo dokonce netestovatelný kód,
způsobený špatnou implementací.~\cite{code_coverage}

Code coverage také monituruje vývoj v čase,
případně popisuje změny,
které by se do aplikace zanesli při připojení kódu ve verzovacím systému.
V rámci systémů rozšiřující verzovací systém lze využívat několik nástrojů,
které code coverage
--- v rámci kontinuální integrace, která je popsána v kapitole~\ref{chap:ci} ---
moniturují a podávají vývojářům zprávy.
Příkladem těchto služeb může být CodeCov či Coveralls.
Každý soubor či složka jsou ohodnoceny procenty,
které značí poměr otestovaného a neotestovaného kódu.
Služba pracující s code coverage při každé obdržené dávce kódu spustí
průzkum kódu aplikace a porovná jej se stávající verzí.~\cite{code_coverage}

\subsection{Test driven development}
\label{sec:tdd}

\emph{Test driven development} (dále jen TDD) je metoda pro vývoj aplikací tím
způsobem,
že se začíná psaním testů,
které je následováno psaním produkčního kódu,
které testy splňuje~\cite{tdd}.
Dle~\cite{tdd} má TDD tři jednoduchá pravidla:

\begin{enumerate}
    \item Není dovoleno psát žádný produkční kód kromě toho,
    který opravuje selhávající test.
    \item Není dovoleno psát více unit testů než je dostačující k selhání;
    a selhání kvůli kompilaci jsou také selhání.
    \item Není dovoleno psát více produkčního kódu,
    než je dostatečné pro úspěšné provedení unit testu.
\end{enumerate}

Nejdříve je tedy napsán unit test pro funkcionalitu,
kterou chce vývojář přidat~\cite{tdd}.
Následně vývojář napíše jen tolik kódu,
kolik je třeba k opravě daného testu a celý proces se znovu opakuje~\cite{tdd}.
Celá myšlenka stojí tedy na tom,
že vývojář může psát pouze kód,
který opravuje selhávající test~\cite{tdd}.
\emph{\uv{Now most programmers,
when they first hear about this technique, think:
\buv{This is stupid.}
\buv{It's going to slow me down,
it's a waste of time and effort,
It will keep me from thinking,
it will keep me from designing,
it will just break my flow.}
However,
think about what would happen if you walked in a room full of people working
this way.
Pick any random person at any random time.
A minute ago, all their code worked.}}~\cite{tdd}
 
Jak již bylo zmíněno,
dle výzkumu~\cite{testing_quality} je navíc tento přístup velmi efektivní,
jelikož vede k vyšší kvalitě kódu.
Navíc,
dodržováním těchto tří pravidel přesně víme,
jak volat určité rozhraní,
jelikož pro toto rozhraní existuje test~\cite{tdd}.
Kdokoli potřebuje znát cokoli o aplikaci,
existuje test,
který tuto činnost popisuje~\cite{tdd}.

\subsection{Kontinuální integrace}
\label{chap:ci}

Jakákoli aplikace se běžně skládá z mnoha částí
a obsahuje mnoho řádků kódu,
zatímco do sebe musí vše dokonale zapadat
a i nejmenší chyba může způsobit selhání sestavení aplikace.
Při přidávání nových funkcionalit nastává velké riziko,
že aplikace obsahuje chybu.
Při snaze připojení změn pomocí nástrojů rozšiřující verzovací systémy
tak proto nastává příležistost pro CI,
která provede sestavení aplikace a spustí automatizované testy.
V tomto okamžiku dostane vývojář skrze rozšiřující aplikace indikaci,
zda změny kód rozbijí,
nebo zda aplikace zůstane v funkční.~\cite{ci}

Použití CI je výhodné především při práci v týmu.
Právě díky zmíněným výhodám lze připojovat změny rychle a bez obav o zanesení
jinak skrytých chyb.
Tyto nástroje jsou buď zabudované do nástrojů rozšiřujících verzovací systémy
jako jsou GitHub a jejich GitHub Actions či GitLab a jejich GitLab CI,
nebo nezávislé nástroje,
jako jsou Travis CI či CircleCI.~\cite{ci}

\subsection{Zhodnocení}

Psát testy je velmi důležité pro každý softwarový produkt.
Zatímco psaní widget testů a integračních testů může být zdlouhavé
a mohou se často měnit požadavky,
psaní unit testů je velmi důležité a dalo by se říci,
že i nutností.
V praktické práci budou tedy využity zejména unit testy,
které pomáhají testovat samotnou funkcionalitu daných částí
a zaručují,
že stanovené rozhraní se nemění.
Tyto testy nadále budou spouštěny v rámci code coverage,
což je sice nejistý indikátor,
ale i tak je dobré jej udržovat co nejvýše.
Testy a code coverage je vhodné spouštět automaticky pomocí kontinuální
integrace.
TDD je skvělá metoda,
která má mnoho benefitů,
jak bylo popsáno v kapitole~\ref{sec:tdd}.
Aplikace této metody ovšem vyžaduje spoustu zkušeností
v rámci několika měsíců,
proto bude tento přístup při vývoji pouze upřednostňován.
