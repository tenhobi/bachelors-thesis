\section{Databáze}

Aby aplikace mohla uchovávat data napříč aplikacemi,
musí využívat některé databázové řešení.
Databáze je souhrn informací,
které jsou vhodně uspořádány tak,
že jsou snadno dostupné a~udržovatelné.
Oproti jednomuchému ukládání dat do souboru na disku mají databáze hodně výhod.
Data lze \mbox{vyhledávat} pomocí speciálních dotazů,
lze je slučovat pomocí spojovacích funkcí,
jsou tolerantní k~chybám
a~může je používat více uživatelů najednou.
A~to všechno při zachování optimální rychlosti.~\cite{database}

Dle~\cite{sql_nosql} existuje několik typů databází a~přístupů k~datům,
přičemž největší rozdíly jsou v~tom,
jak ukládají který typ dat,
a~jak je k~těmto datům umožněno přistupovat.
\emph{\uv{Relational databases are structured,
while non-relational databases are document-oriented and
distributed.}}~\cite{sql_nosql}

Zatímco Structured Query Language (dále jen SQL) databáze 
byly\linebreak donedávna
preferovaným způsobem databází,
trendem posledních let jsou Non~SQL (dále jen NoSQL) databáze~\cite{sql_nosql}.
Popis těchto typů databází,
včetně výhod a~nevýhod jednotlivého řešení,
je uveden v~následujících sekcích.

\subsection{SQL}

Data v~SQL databázích jsou ukládány strukturovaně,
čímž tyto databáze dosahují největších výhod.
Výhodou je také jejich široká podpora a~snadné řešení chyb.
Principem těchto databází jsou vlastnosti,
které shrnuje ACID (atomičnost, koexistence, izolace, trvanlivost).
Těmito vlastnostmi databáze redukuje anomálie a~chyby způsobené transakcemi.
Díky dlouhé době,
za kterou se SQL databáze již používaly,
existuje spousta nástrojů a~doplňků,
které usnadňují vývoj a~řešení případných chyb.
Nevýhodou je však náročnější škálování databáze při vývoji aplikace.
Příkladem je například databáze MySQL.~\cite{sql_nosql}

\subsection{NoSQL}

Data v~NoSQL databázích jsou většinou nestrukturovány.
Tím získávají tyto databáze velký náskok co se flexibility dat a~rychlosti
přístupu týče.
Tyto databáze je vhodné vnímat především jako kolekce,
které obsahují dokumenty. 
NoSQL databáze jsou vhodné pro ukládání velkého množství nestrukturovaných dat,
u~kterých není nutné předem definovat jejich podobu.
Nevýhodou je však relativní novost tohoto přístupu,
což znamená i~menší počet nástrojů,
jako jsou nástroje k~analýze a~řešení problémů.
Tyto databáze většinou používají vlastní dotazovací jazyk či jiné standardizace,
což může vést k~problémům při přechodu na jinou databázi~\cite{sql_nosql}.
Příkladem je například databáze Cloud Firestore~\cite{cloud_firestore}.

\subsection{Firebase: Cloud Firestore}

Firebase je projekt,
který pomáhá vývojářům s~mobilními a~webovými aplikacemi.
Díky Firebase mohou být aplikace tvořeny rychleji,
bez času stráveného budováním funkcionalit jako analytika a~databáze.
Tyto produkty jsou tvořeny na infrastruktuře Google
a~jsou využívány společnostmi jako jsou The~New York Times,
Duolingo, Alibaba a~další.~\cite{firebase}

Cloud Firestore je flexibilní NoSQL databáze dostupná v~projektu \mbox{Firebase}.
Tato databáze lehce spolupracuje s~ostatními produkty Firebase,
jako jsou autentizace, správa práv nebo automatické aktualizace dat.
Spolu s~databází je poskytována řada balíčků pro vývoj mobilních a~webových
aplikací,
které poskytují přátelské rozhraní k~vývoji s~touto databází.
Dostupná je také funkce offline podpory,
která automaticky ukládá často používané data,
které je možno používat i~bez internetového spojení.
Data je možné filtrovat pomocí speciálních dotazů na kolekce i~dokumenty
a~je možné také tyto dotazy kombinovat a~řadit.~\cite{cloud_firestore}

\subsection{Zhodnocení}

Vývoj herní aplikace je ideální pro využití NoSQL databází.
V~takovéto aplikaci se najde plno příležitostí,
kde lze využít rysů těchto databází,
kde by naopak rysy SQL databází působily zbytečné obtíže.
I~když nedisponují vlastnostmi ACID,
je využití tohodo typu databáze,
konkrétně databáze Cloud Firestore,
nejvhodnější pro vývoj praktické části práce,
proto bude použita právě databáze Cloud Firestore.
