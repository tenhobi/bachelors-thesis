\section{Android}

Pro sestavení aplikace na platformě Android je nutné stáhnout a přidat soubor
\mintinline{shell}|android/app/google-services.json|,
který obsahuje nastavení služby databáze Cloud Firestore.
Aplikaci je vhodné přidat vlastní ikonu,
která se uloží do složky \mintinline{shell}|android/app/src/main/res/|,
Pro samotnou distribuci je nutné sestavit aplikaci
s podepsaným certifikačním klíčem,
jinak aplikace nepůjde publikovat v obchodě Google Play. 
Tento klíč se vygeneruje pomocí nástroje \mintinline{shell}|keytool|
a musí se na něj odkázat ze souboru \mintinline{shell}|android/key.properties|.
V souboru
\mintinline{shell}|android/app/src/main/AndroidManifest.xml|
lze také nastavit položku
\mintinline{shell}|application android:label|
pro nastavení názvu aplikace.
V souboru
\mintinline{shell}|android/app/build.gradle|
lze naopak nastavit údaje pro distribuci,
konkrétně položka \mintinline{shell}|applicationId|
obsahuje unikátní identifikátor aplikace
a položky \mintinline{shell}|versionCode| a \mintinline{shell}|versionName|
obsahují název a číslo konkrétního vydání aplikace.
\cite{flutter_deploy_android}

Příkaz \mintinline{shell}|flutter build appbundle| sestaví výslednou aplikaci
pro distribuci,
kde \emph{app bundle} je preferovaný formát pro distribuci,
oproti klasickému formátu \emph{apk}.
Následně je možné distribuovat aplikaci přes rozhraní Google Play Console
do obchodu Google Play.
