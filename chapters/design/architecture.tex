\section{Architektura}

Návrh architektury je zásadní krok při vývoji aplikace.
Nevhodně zvolená architektura může znemožnit snadné rozšíření aplikace
a~její údržbu.
Při návrhu architektury je proto dbán zřetel na jednoduchost a~univerzálnost.
Návrh se netýká jen aplikace na nejvyšší úrovni,
ale i~dílčích částí. 
V~následujících sekcích jsou popsány navrhované řešení herní smyčky,
dělení aplikace na oddělené moduly a~využití knihovny Bloc v~aplikaci.

\subsection{Herní smyčka}

Pro herní aplikaci je důležitá herní smyčka,
která popisuje samotné fungování hry a~věci,
které je nutné mezi zařízeními sdílet.
Ve hře je potřeba udržovat informace o~splněných herních modulech,
splněných akcích a~o~dalších informacích o~hře.
Některé informace, jako jsou informace o~herních modulech,
mohou zůstat na zařízení hráče.
Hráč s~manuálem nepotřebuje,
a~neměl by,
vědět všechny informace.
Tyto informace mohou být v~rámci týmové komunikace předány ústně.
Díky tomu je možné návrh zjednodušit a~tyto informace nesdílet
a~udržovat je pouze v~zařízení hráče s~herními moduly.

Důležité informace pro oba hráče jsou informace o~splnění všech modulů,
což znamená ukončení hry s~výhrou.
Dále informace o~překročení časového limitu,
což znamená ukončení hry s~prohrou.
A~také informace o~splnění akce na daném zařízení,
resp. splnění akce na obou zařízeních,
což vede k~odblokování obrazovek a~možnosti pokračovat ve hře.
Mezi zařízeními budou tedy sdíleny pouze zmíněné informace --
dokončení hry, překročení času a~společné splnění akce.
Jelikož není potřeba sdílet většinu informací,
nejsou na hráče kladeny nároky na rychlost spojení.

Samotné nastavení mise obsahuje seznam modulů,
které jsou v~misi využity.
Pro každý modul je zaznamenáno konkrétní nastavení,
které se aplikuje do hry jak do manuálu, tak i~do podmínek pro splnění modulu.
Dále obsahuje informaci o~čase k~dokončení hry.

\subsection{Moduly}

Aplikace bude dle architektury Clean Architecture rozdělena
na úrovni modulů na tři části.
Tyto moduly obsahují zvlášť prezentační, aplikační a~datovou vrstvu.
První modul se zaměřuje na obrazovky mimo hru a~její nastavení
a~řeší autentizaci a~přihlášení.
Druhý modul se zaměřuje na nastavení hry a~její spuštění a~řeší propojení hráčů.
Třetí modul se zaměřuje na hru samotnou a~řeší její průběh.
Kromě těchto specifických modulů je v~obecném kontextu řešeno nastavení
aplikace,
které zmíněné moduly využívají.

\subsection{Bloc}

Prezenční vrstvu s~vrstvou aplikační spojují kontrolery knihovny Bloc.
Ty jsou reprezentovány třídami \mintinline{Dart}{Bloc} a~udržují a~mění stav
dané části.
Sdílená třída \mintinline{Dart}{Bloc} je například pro nastavení,
které je nutné sdílet mezi všemi moduly.

Modul home obsahuje třídu \mintinline{Dart}{Bloc} pro autentizaci a~přihlášení.
Tyto třídy jsou rozděleny,
protože zatímco autentizace řeší přepínání obrazovek podle stavu uživatele,
přihlášení řeší přihlášení samotné na dané obrazovce.
Protože každý stav dané třídy odpovídal pouze tomu,
co by měl obsahovat,
tyto třídy nelze spojit.
Příkladem je obrazovka přihlašování,
která se zobrazí pokud stav autentizace je \mintinline{Dart}{unauthenticated},
avšak může také zobrazit chubu,
která je reprezentována stavem přihlašování \mintinline{Dart}{failure}.
Pokud by tento stav byl v~rámci autentizace,
byl by ztracen kontext.

Modul lobby obsahuje třídu \mintinline{Dart}{Bloc} reprezentující stav
vytváření hry.
Jelikož se v~tomto modulu nemění více stavů,
je přítomen pouze jeden.
Tomuto stavu se však průběžně mění data jako zvolená mise,
přítomní hráči a~zda jsou uživatelé připraveni.

Obdobně funguje třída \mintinline{Dart}{Bloc} modulu game,
která však obsahuje stav i~pro obrazovku ukončené hry se statistikami.
Navíc má i~stav pro hráče s~manuálem,
díky kterému se zobrazí informace namísto herních modulů.
Třída pro hráče s~herními moduly shromažďuje shromažďuje data o~jednotlivých
modulech a~jejich řešení.
Tento modul obsahuje ještě speciální třídu \mintinline{Dart}{Bloc} pro akce,
které mohou překrývat obrazovku samotné hry.
